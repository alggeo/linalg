\documentclass[a4,11pt]{article}

\usepackage{exercises}
\usepackage{macros}
\usepackage{mathtools}

\usepackage{graphicx}
\usepackage{ngerman}

\vltitel{Lineare Algebra 2}
\dozent{\small{Christian Haase}}
\assistent{\small{Jan Marten Sevenster}}
\tutoren{\small{%
    Theresa Graeber \\[-1ex] Eva Schinzel}}

\semester{Sommersemester 2023%
  % \raisebox{-10mm}[0pt][0pt]{%
  %   \parbox{0pt}{\includegraphics[width=27mm]{../../2015-ana1-L/Vorlesungsmaterial/ana1QR}}}
}

\DeclareMathOperator{\End}{End}

\begin{document}
\vspace*{-17mm}
{
\kopf
}
% \vspace*{-5mm}
% \enlargethispage*{25mm}

\newcounter{chapter}
\ueblatt{6}{ Montag, 05.~Juni 2023 um 10h00 (ein PDF, jede Aufgabe auf neuer Seite beginnen)}

Diese Woche werden Sie statt Hausaufgaben ein Schreiblabor machen. Die Quizfrage sollen Sie nicht einreichen und wird nicht bewertet oder korrigiert, weil es keine Zentralübung gibt diese Woche, sie dient als "Ubung und kann am sp"ateren Zeitpunkt besprochen werden.

\bigskip

\begin{aufgabe}[Scheiblaboraufgabe]
Sei $F$ ein Endomorphismus des Vektorraums $V$ "uber dem K"orper $K$,
und sein $v \in V$.
% Dann ist der von $v$ erzeugte $F$-zyklische Unterraum von $V$ (Notation span??)
% $$ U_F(v) := span(v, F(v), F(F(v)), \ldots) \,.$$

\newcommand{\bonusitem}{\item\hspace*{-2.4mm}*\ }

  \begin{enumerate}
  \item Rufen Sie sich die Definition des von $v$ erzeugten
    $F$-zyklischen Unterraums $U_F(v)$ ins Ged"achtnis.
  \item "Uberzeugen Sie sich nochmal, dass $U_F(v)$ ein
    $F$-invarianter Unterraum ist.
  \item Wenn $V$ endlichdimensional ist, und $n := \dim U_F(v)$, warum
    ist dann \\ $v, F(v), F(F(v)), \ldots, F^{n-1}(v)$ eine
    Basis von $U_F(v)$?
  \item Seien $a \in K^n$ der Koordinatvektor von $F^n(v)$ in dieser Basis. Explizit bedeutet das, dass wir
  \[
  F^n(v) = a_0 v + \cdots a_{n-1} F^{n-1}(v) 
  \]
  schreiben können, wegen des vorherigen Teils. Jetzt ist $a = (a_0, \dots , a_{n-1})^T$.
    Bezeichne mit $\bar F \colon U_F(v) \to U_F(v)$ die 
    Einschr"ankung von $F$ auf $U_F(v)$.
    
    Beschreiben Sie die Darstellende Matrix von $\bar F$ bez"uglich
    dieser Basis mithilfe der Koeffizienten $a$.
  \item Dr"ucken Sie $P_{\bar F}$ mithilfe der Koeffizienten $a$ aus.
  \item Zeigen Sie, dass $P_{\bar F}(\bar F(v)) = 0$.
  \bonusitem Warum gen"ugen diese "Uberlegungen, um den Satz von
    Cayley-Hamilton zu beweisen?
\end{enumerate}

\end{aufgabe}

\newpage
\bigskip
{\Large \bfseries Wiederholungsaufgaben}
\bigskip

\begin{aufgabe}
Wir nehmen drei Polynome in $\Q[x]$
\begin{align*}
p &= x^8 + 3x^7 + 2x^6 + x^5 - x^4 - 3x^3 - 2x^2 - x\\
q &= 2x^7 + 2x^6 + x^4 - 2x^3 - 2x^2 - 1\\
r &= x^5 + x^4 - x^3 - 1,
\end{align*}
\begin{enumerate}
\item
  Bestimmen Sie ein Polynom $s$, so dass $(p,q) = (s)$
  und $a,b$, so dass $ap+bq=s$.
\item
  Bestimme ein Polynom $s$, so dass $(p,q,r) = (s)$.
 \end{enumerate}
\end{aufgabe}

\begin{aufgabe}

Es sei $R$ ein kommutativer Ring und $I, J \subset R$ Ideale. Dann
definieren wir die Summe, das Produkt und den Durchschnitt von $I$ und
$J$ wie folgt.
\begin{align*}
I + J &= \{ i + j \mid i \in I, j \in J \}\\
IJ &= \left\{ \sum_{k = 1}^n i_k j_k \mid i_k \in I, j_k \in J, k \geq 0 \right\}\\
I \cap J &= I \cap J.
\end{align*}
Zeigen Sie, dass diese drei Mengen Ideale von $R$ sind. Zeigen Sie
weiter, dass $IJ \subseteq I \cap J$ und geben Sie ein Ring $R$ und
Ideale $I, J \subset R$ an, f"ur die die umgegehrte Inklusion nicht gilt.  
\end{aufgabe}


\newpage
\bigskip
{\Large \bfseries Quizfrage}
\bigskip

Sind $\left<-,-\right>_1$ und $\left<-,-\right>_2$ zwei Skalarprodukten auf einen $\mathbb{C}$ Vektorraum $V$, bestimmen Sie dann, welche der folgenden Funktionen 
\begin{align*}
\left<-,-\right>_{1 + 2} &: V \times V \rightarrow \mathbb{C} : (v, w) \mapsto \left<v,w\right>_1 + \left<v,w\right>_2\\
\left<-,-\right>_{1 - 2} &: V \times V \rightarrow \mathbb{C} : (v, w) \mapsto \left<v,w\right>_1 - \left<v,w\right>_2
\end{align*}
wieder ein Skalarprodukt auf $V$ definieren.

\end{document}

%%% Local Variables: 
%%% mode: latex
%%% End: 
