\documentclass[a4,11pt]{article}

\usepackage{./exercises}
\usepackage{./macros}
\usepackage{mathtools}
\usepackage{tikz}

\usepackage{graphicx}
\usepackage{ngerman}

\vltitel{Lineare Algebra 2}
\dozent{\small{Christian Haase}}
\assistent{\small{Jan Marten Sevenster}}
\tutoren{\small{%
    Theresa Graeber \\[-1ex] Eva Schinzel}}

\semester{Sommersemester 2023%
  % \raisebox{-10mm}[0pt][0pt]{%
  %   \parbox{0pt}{\includegraphics[width=27mm]{../../2015-ana1-L/Vorlesungsmaterial/ana1QR}}}
}

\DeclareMathOperator{\End}{End}
\DeclareMathOperator{\Span}{span}
% \DeclareMathOperator{\ker}{ker}
\DeclareMathOperator{\im}{im}
\newcommand{\bonusitem}{\item\hspace*{-2.4mm}*\ }


\begin{document}
\vspace*{-17mm}
{
\kopf
}
% \vspace*{-5mm}
% \enlargethispage*{25mm}

\newcounter{chapter}
\ueblatt{13}{ Montag, 24.~Juli 2023 um 10h00 -- Bonusblatt}


\begin{aufgabe}[4 Punkte]
  Wir beweisen nochmal, dass jede endlichdimensionale Darstellung
  $\rho \colon G \to \operatorname{GL}(V)$ einer endlichen Gruppe $G$ in irreduzible
  Darstellungen zerf"allt.
\begin{enumerate}
\addtocounter{enumi}{-1}
\bonusitem
Das zentrale Lemma garantiert f"ur jeden $G$-invarianten Unterraum $U
\subseteq V$ einen $G$-invarianten Unterraum $U' \subseteq V$, so dass
$V = U \oplus U'$ als $G$-Darstellungen.

Folgern Sie aus dieser Aussage, dass jede endlichdimensionale
$G$-Darstellung in irreduzible zerf"allt.
\item Gegeben $U \subseteq V$ wie oben, konstruieren Sie einen
  komplement"aren Unterraum $W \subseteq V$ zusammen mit einer
  Projektion $\pi \colon V \to U$ mit Kern $W$ und Bild
  $U$. ("`Projektion"' hei"st, $\pi \circ \pi = \pi$, insbesondere
  $\pi|_U = \id_U$.)
\item Definieren Sie $\hat \pi \colon V \to V$ als
  $$
  \hat \pi = \frac{1}{|G|} \sum_{g \in G} \rho(g) \circ \pi \circ \rho(g^{-1}) \,.
  $$
  Zeigen Sie, dass $\hat \pi$ eine $G$-invariante Projektion mit Bild $U$ ist.
\item Sei $U'$ das Bild von $\hat \pi$. Zeigen Sie, dass $V = U \oplus
  U'$ als $G$-Darstellungen.
  \end{enumerate}
\end{aufgabe}

\begin{aufgabe}[4 Punkte]
Es sei $G$ eine Gruppe und $g,h \in G$ Elementen aus $G$, dann hei"st $g$ zu $h$ konjugiert, wenn es einen $c \in G$ gibt, so dass $g = chc^{-1}$.\begin{enumerate}
\item
Zeigen Sie, dass die Konjugationsrelation ($g \sim h$ genau dann, wenn $g$ zu $h$ konjugiert ist) eine "Aquivalenzrelation ist. Die "Aquivalenzklassen dieser Relation werden Konjugationsklassen genannt.
\item
Es sei nun $G$ eine endliche Gruppe und $\rho$ eine Darstellung von $G$. Zeigen Sie, dass wenn $g$ und $h$ in derselben Konjugationsklasse enthalten sind, dass dann
\[
\text{Spur}(\rho(g)) = \text{Spur}(\rho(h)) 
\]
gilt.
\item
Wir nehmen die Permutationsgruppe auf $4$ Elementen:
\[
S_4 = \{ \sigma : \{ 1,2,3,4 \} \to \{ 1,2,3,4 \} \mid \sigma \text{ ist bijektiv } \}.
\]
Listen Sie die Konjugationsklasse dieser Gruppe der $4! = 24$ Elementen auf.
\item
Eine wichtige Darstellung der $S_4$ ist die Permuationsdarstellung. Sie permutiert die Basisvektoren eines vierdimensionalen $K$-Vektorraums.
\[
\pi : S_4 \to \operatorname{GL}(\operatorname{span}(v_1, v_2, v_3, v_4)) : \pi(\sigma)(v_i) = v_{\sigma(i)}.
\]
Gen"ugt die Spurabbildung hier um Konjugationsklasse von $S_4$ voneinander zu unterscheiden? Also gilt
\[
\text{Spur}(\pi(\sigma_1)) = \text{Spur}(\pi(\sigma_2)) \Leftrightarrow \sigma_1 \sim \sigma_2 ?
\]
\end{enumerate}

\end{aufgabe}

\begin{aufgabe}[4 Punkte]
Ein Homomorphismus zwischen Darstellungen $\rho : G \to \operatorname{GL}(V)$ und $\tau : G \to \operatorname{GL}(W)$ von $G$ ist eine Abbildung $T \in \operatorname{Hom}(V,W)$, sodass
\[
T \circ \rho(g) = \tau(g) \circ T \ \text{ f"ur alle } g \in G.
\]
Die Menge der Homomorphismen zwischen Darstellungen $\rho$ und $\tau$, bezeichnen wir mit $\operatorname{Hom}(\rho, \tau)$.
Zeigen Sie, dass f"ur zwei Homomorphismen $T_1 \in \operatorname{Hom}(\rho, \tau)$ und $T_2 \in \operatorname{Hom}(\tau, \upsilon)$, jeweils zwischen Darstellungen von $G$, ihre Verkettung $T_2 \circ T_1$ wieder ein Homomorphismus zwischen Darstellungen von $G$ ist.
\end{aufgabe}

\end{document}

%%% Local Variables: 
%%% mode: latex
%%% End: 
