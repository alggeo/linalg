\documentclass[a4,11pt]{article}

\usepackage{./exercises}
\usepackage{macros}
\usepackage{mathtools}
\usepackage{tikz-cd}

\usepackage{graphicx}
\usepackage{ngerman}

\vltitel{Lineare Algebra 2}
\dozent{\small{Christian Haase}}
\assistent{\small{Jan Marten Sevenster}}
\tutoren{\small{%
    Theresa Graeber \\[-1ex] Eva Schinzel}}

\semester{Sommersemester 2023%
  % \raisebox{-10mm}[0pt][0pt]{%
  %   \parbox{0pt}{\includegraphics[width=27mm]{../../2015-ana1-L/Vorlesungsmaterial/ana1QR}}}
}

\DeclareMathOperator{\End}{End}

\begin{document}
\vspace*{-17mm}
{
\kopf
}
% \vspace*{-5mm}
% \enlargethispage*{25mm}

\newcounter{chapter}
\ueblatt{7}{ Montag, 12.~Juni 2023 um 10h00}

% \begin{aufgabe}[4 Punkte]
% Wir setzen 
% \[
% v_1 = \begin{pmatrix}1\\-1\\2\end{pmatrix}, \ v_2 = \begin{pmatrix}-1\\0\\0\end{pmatrix}, \ v_3 = \begin{pmatrix}2\\2\\2\end{pmatrix} \in \mathbb{R}^3.
% \]
% Bestimmen Sie einen Skalarprodukt auf $\mathbb{R}^3$ in Bezug auf den $v_1$, $v_2$ und $v_3$ ein Orthomornalsystem formen.
% \end{aufgabe}

\begin{aufgabe}[4 Punkte]
  Betrachte den % $2$-dimensionalen
  $\Q$-Vektorraum
  $$
  \Q\left[\sqrt{-5}\right] :=  \left\{ a + b\,i\,\sqrt{5} \in \C \suchthat a,b \in \Q \right\} \,.
    % \span(1, \sqrt{-5}) \subset \C \,.
  $$
  Auf diesem Vektorraum ist durch $q(z) := z \bar z$ eine quadratische Form definiert.

  Bestimmen Sie die dazugeh"orige symmetrische Bilinearform
  $s$.

  W"ahlen Sie eine Basis von $\Q\left[\sqrt{-5}\right]$ und
  bestimmen Sie die darstellende Matrix von $s$ bez"uglich Ihrer
  Basis.
\end{aufgabe}

\begin{aufgabe}[4 Punkte]
Wir definieren eine Abbildung 
$$
\begin{array}{rccc}
  \left<-,-\right>_{F} : & \text{Mat}(m \times n; \mathbb{C})
                           \times\text{Mat}(m \times n; \mathbb{C}) &
                                                                      \to
  & K \\
                         & (A,B) & \mapsto & \text{Spur}(\transpose A \overline{B}).
\end{array}
$$

\begin{enumerate}
\item Zeigen Sie, dass $\left<-,-\right>_{F}$ ein Skalarprodukt auf
  $\text{Mat}(m \times n; \mathbb{C})$ definiert.
\item Betrachten wir die Vektorr"aume $\mathbb{C}^n$ und
  $\mathbb{C}^m$ mit ihren Standardbasen $\mathcal{B}_n$ und
  $\mathcal{B}_m$, dann ist der Vektorraum $\text{Mat}(m \times n;
  \mathbb{C})$ "ublicherweise isomorph zum Vektorraum
  $\text{Hom}_\mathbb{C}(\mathbb{C}^n, \mathbb{C}^m)$. Nehmen wir
  jetzt verschiedene Basen $\mathcal{B}'_n$ und $\mathcal{B}'_m$, dann
  erhalten wir einen zweiten Isomorphismus
\[
\text{Hom}_\mathbb{C}(\mathbb{C}^n, \mathbb{C}^m) \to \text{Mat}(m
\times n; \mathbb{C}) : f \mapsto
M_{\mathcal{B}'_n}^{\mathcal{B}'_m}(f).
\]
Zeigen Sie, wenn $\mathcal{B}'_n$ und $\mathcal{B}'_m$
Orthonormalbasen sind, dass dann
\[
\left<M_{\mathcal{B}_n}^{\mathcal{B}_m}(f),M_{\mathcal{B}_n}^{\mathcal{B}_m}(g)\right>_{F}
=
\left<M_{\mathcal{B}'_n}^{\mathcal{B}'_m}(f),M_{\mathcal{B}'_n}^{\mathcal{B}'_m}(g)\right>_{F}
\]
f"ur alle $f, g \in \text{Hom}_\mathbb{C}(\mathbb{C}^n, \mathbb{C}^m)$ gilt.
\end{enumerate}
\end{aufgabe}

\begin{aufgabe}[4 Punkte]
Benutzen Sie die Cauchy–Bunyakovsky–Schwarz Ungleichung um die Ungleichung
\[
\frac{(a_1 + \dots +  a_n)^2}{b_1 + \dots + b_n} \leq \frac{a_1^2}{b_1} + \dots + \frac{a_n^2}{b_n}
\]
zu beweisen f"ur alle $(a_i)_{i = 1}^n \in \mathbb{R}^n$ und $(b_i)_{i=1}^n \in \mathbb{R}^n_{\geq 0}$.
\end{aufgabe}

\end{document}

%%% Local Variables: 
%%% mode: latex
%%% End: 
