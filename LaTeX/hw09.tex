\documentclass[a4,11pt]{article}

\usepackage{./exercises}
\usepackage{./macros}
\usepackage{mathtools}
\usepackage{tikz-cd}

\usepackage{graphicx}
\usepackage{ngerman}

\vltitel{Lineare Algebra 2}
\dozent{\small{Christian Haase}}
\assistent{\small{Jan Marten Sevenster}}
\tutoren{\small{%
    Theresa Graeber \\[-1ex] Eva Schinzel}}

\semester{Sommersemester 2023%
  % \raisebox{-10mm}[0pt][0pt]{%
  %   \parbox{0pt}{\includegraphics[width=27mm]{../../2015-ana1-L/Vorlesungsmaterial/ana1QR}}}
}

\DeclareMathOperator{\End}{End}
\DeclareMathOperator{\Span}{span}
% \DeclareMathOperator{\ker}{ker}
\DeclareMathOperator{\im}{im}
\newcommand{\bonusitem}{\item\hspace*{-2.4mm}*\ }


\begin{document}
\vspace*{-17mm}
{
\kopf
}
% \vspace*{-5mm}
% \enlargethispage*{25mm}

\newcounter{chapter}
\ueblatt{9}{ Montag, 26.~Juni 2023 um 10h00}


\begin{aufgabe}[4 Punkte]
Es sei $M \in \text{Mat}(n \times n; \mathbb{R})$ eine symmetrische Matrix. F"ur eine nichtleere Teilmenge 
\[
\emptyset \neq I = \{ i_1 < i_2 < \cdots < i_k \} \subseteq \{ 1, \dots, n\}
\]
setzen wir
\[
d_I = \det \begin{pmatrix} m_{i_1 i_1} & \cdots & m_{i_1 i_k} \\ \vdots & \ddots & \vdots \\ m_{i_k i_1} & \cdots & m_{i_k i_k} \\ \end{pmatrix}.
\]
\begin{enumerate}
\item
Zeigen Sie, dass $M$ genau dann positiv simi-definit ist, wenn $d_I \geq 0 \ \forall  \ \emptyset \neq I \subset \{ 1, \dots, n\}$.
\item
Wieviele Determinanten m"ussen also ausgerechnet werden, um auf diese Weise zu "uberpr"ufen, ob $M$ positiv semi-definit ist?
\end{enumerate}
\end{aufgabe}

\begin{aufgabe}[4 Punkte]
  \begin{enumerate}
  \item Geben Sie ein Beispiel eines nicht-trivialen Endomorphismus
    $F$ eines euklidischen Vektorraums $V$, sodass f"ur alle $v \in V$
    \begin{equation}
      \tag{$**$}
      \langle F(v), v \rangle = 0 \,.
    \end{equation}
  \bonusitem Schlagen Sie die verschiedenen Charakterisierungen nilpotenter
  Endomorphismen nach.
  \bonusitem Was war nochmal ein $F$-zyklischer Unterraum?
  \item Sei nun $F$ ein Endomorphismus eines endlichdimensionalen,
    unit"aren Vektorraums, der ($**$) erf"ullt. Zeigen Sie, dass $F=0$ sein muss.
  \end{enumerate}
\end{aufgabe}

\begin{aufgabe}[4 Punkte]
  Sei $V$ ein euklidischer oder unit"arer Vektorraum und $F
  \in \End(V)$ ein Endomorphismus, der einen adjungierten
  Endomorphismus $F^*$ hat. Zeigen Sie
  $$
  \ker F^* \subseteq (\im F)^\perp \qquad \text{ und } \qquad \im F^*
  \subseteq (\ker F)^\perp \,.
  $$
  Wenn $V$ endlichdimensional ist, folgern Sie, dass in beiden F"allen
  Gleichheit gilt.
\end{aufgabe}

\footnotetext{*: Ggf.\ im Tutorium besprechen.}


\end{document}

%%% Local Variables: 
%%% mode: latex
%%% End: 
