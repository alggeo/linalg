\documentclass[a4,11pt]{article}

\usepackage{exercises}
\usepackage{macros}
\usepackage{mathtools}
\usepackage{tikz-cd}

\usepackage{graphicx}
\usepackage{ngerman}

\vltitel{Lineare Algebra 2}
\dozent{\small{Christian Haase}}
\assistent{\small{Jan Marten Sevenster}}
\tutoren{\small{%
    Theresa Graeber \\[-1ex] Eva Schinzel}}

\semester{Sommersemester 2023%
  % \raisebox{-10mm}[0pt][0pt]{%
  %   \parbox{0pt}{\includegraphics[width=27mm]{../../2015-ana1-L/Vorlesungsmaterial/ana1QR}}}
}

\DeclareMathOperator{\End}{End}
\DeclareMathOperator{\Span}{span}
\newcommand{\bonusitem}{\item\hspace*{-2.4mm}*\ }


\begin{document}
\vspace*{-17mm}
{
\kopf
}
% \vspace*{-5mm}
% \enlargethispage*{25mm}

\newcounter{chapter}
\ueblatt{8}{ Montag, 19.~Juni 2023 um 10h00}


\begin{aufgabe}[4 Punkte]
pos semi-definit wenn alle Hauptminoren $\ge 0$
\end{aufgabe}

\begin{aufgabe}[4 Punkte]
  \begin{enumerate}
  \item Geben Sie ein Beispiel eines nicht-trivialen Endomorphismus
    $F$ eines euklidischen Vektorraums $V$, sodass f"ur alle $v \in V$
    \begin{equation}
      \tag{$*$}
      \langle F(v), v \rangle = 0 \,.
    \end{equation}
  \item Schlagen Sie die verschiedenen Charakterisierungen nilpotenter
    Endomorphismen nach.
  \item Sei nun $F$ ein Endomorphismus eines unit"aren Vektorraums,
    der ($*$) erf"ullt. Zeigen Sie, dass $F=0$ sein muss.
  \end{enumerate}
\end{aufgabe}

\begin{aufgabe}[4 Punkte]
unit"ar / orthogonal
\end{aufgabe}

\end{document}

%%% Local Variables: 
%%% mode: latex
%%% End: 
