\documentclass[a4,11pt]{article}

\usepackage{exercises}
\usepackage{macros}

\usepackage{graphicx}
\usepackage{ngerman}

\vltitel{Lineare Algebra 2}
\dozent{\small{Christian Haase}}
\assistent{\small{Jan Marten Sevenster}}
\tutoren{\small{%
    Theresa Graeber \\[-1ex] Eva Schinzel}}

\semester{Sommersemester 2023%
  % \raisebox{-10mm}[0pt][0pt]{%
  %   \parbox{0pt}{\includegraphics[width=27mm]{../../2015-ana1-L/Vorlesungsmaterial/ana1QR}}}
}


% https://math.ug/la2-ss21/sec-cayley-hamilton.html


\begin{document}
\vspace*{-17mm}
{
\kopf
}
% \vspace*{-5mm}
% \enlargethispage*{25mm}

\newcounter{chapter}
\ueblatt{3}{ Montag, 15.~Mai 2023 um 10h00 (ein PDF, jede Aufgabe auf neuer Seite beginnen)}

\begin{aufgabe}[4 Punkte]
  Betrachten Sie den Endomorphismus $F$ von $\mathbb{Q}^3$, in der Standardbasis gegeben durch
  \[
  \begin{pmatrix}
  -120 & 164 & -102\\
 19 & -24 & 16\\
 178 & -240 & 151
  \end{pmatrix}
  \]
  Geben Sie zwei verschiedene $F$ trigonalisierende Basen von $\mathbb{Q}^3$ an.
\end{aufgabe}

\begin{aufgabe}[4 Punkte]
  Sei $V$ ein Vektorraum "uber dem K"orper
  $K$. Ein Endomorphismus $F \colon V \to V$ hei"st nilpotent, wenn es
  ein $k \in \Z_{\ge 1}$ gibt, so dass $F^k=0$ die Nullabbildung ist.

\begin{enumerate}
\item Sei nun $V$ endlichdimensional und $F$ nilpotent. Bestimmen Sie
  das charakteristische Polynom $P_F$.
\item Zeigen Sie, dass es eine Basis $B$ von $V$ gibt, so dass
  $M_B(F)$ streng obere Dreiecksform annimmt, d.h., $j \leq i$ impliziert $M_B(F)_{ij} = 0$.
\item Folgern Sie, dass $F^{\dim V} = 0$.
\end{enumerate}

\end{aufgabe}


% Die folgende Aufgabe wiederholt den Beweis von C-H aus der VL.
% (4-6) sind die eigentliche Aufgabe.
% (1-3,7) sind als Lernhinweis gedacht.


\begin{aufgabe}[4 Punkte]
Sei $F$ ein Endomorphismus des Vektorraums $V$ "uber dem K"orper $K$,
und sein $v \in V$.
% Dann ist der von $v$ erzeugte $F$-zyklische Unterraum von $V$ (Notation span??)
% $$ U_F(v) := span(v, F(v), F(F(v)), \ldots) \,.$$

\newcommand{\bonusitem}{\item\hspace*{-2.4mm}*\ }

  \begin{enumerate}
  \bonusitem Rufen Sie sich die Definition des von $v$ erzeugten
    $F$-zyklischen Unterraums $U_F(v)$ ins Ged"achtnis.
  \bonusitem "Uberzeugen Sie sich nochmal, dass $U_F(v)$ ein
    $F$-invarianter Unterraum ist.
  \bonusitem Wenn $V$ endlichdimensional ist, und $n := \dim U_F(v)$, warum
    ist dann \\ $v, F(v), F(F(v)), \ldots, F^{n-1}(v)$ eine
    Basis von $U_F(v)$?
  \item Seien $a \in K^n$ die Koordinaten von $F^n(v)$ in dieser Basis.
    Bezeichne mit \\ $\bar F \colon U_F(v) \to U_F(v)$ die 
    Einschr"ankung von $F$ auf $U_F(v)$.
    
    Beschreiben Sie die Darstellende Matrix von $\bar F$ bez"uglich
    dieser Basis mithilfe der Koeffizienten $a$.
  \item Dr"ucken Sie $P_{\bar F}$ mithilfe der Koeffizienten $a$ aus.
  \item Zeigen Sie, dass $P_{\bar F}(\bar F(v)) = 0$.
  \bonusitem Warum gen"ugen diese "Uberlegungen, um den Satz von
    Cayley-Hamilton zu beweisen?
\end{enumerate}

*: im Tutorium besprechen.
\end{aufgabe}

\end{document}

%%% Local Variables: 
%%% mode: latex
%%% End: 
