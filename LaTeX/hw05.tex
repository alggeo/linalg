\documentclass[a4,11pt]{article}

\usepackage{exercises}
\usepackage{macros}
\usepackage{mathtools}

\usepackage{graphicx}
\usepackage{ngerman}

\vltitel{Lineare Algebra 2}
\dozent{\small{Christian Haase}}
\assistent{\small{Jan Marten Sevenster}}
\tutoren{\small{%
    Theresa Graeber \\[-1ex] Eva Schinzel}}

\semester{Sommersemester 2023%
  % \raisebox{-10mm}[0pt][0pt]{%
  %   \parbox{0pt}{\includegraphics[width=27mm]{../../2015-ana1-L/Vorlesungsmaterial/ana1QR}}}
}

\DeclareMathOperator{\End}{End}

\begin{document}
\vspace*{-17mm}
{
\kopf
}
% \vspace*{-5mm}
% \enlargethispage*{25mm}

\newcounter{chapter}
\ueblatt{5}{ Montag, 29.~Mai 2023 um 10h00 (ein PDF, jede Aufgabe auf neuer Seite beginnen)}

\begin{aufgabe}[4 Punkte]

Erinneren Sie sich an die lineare Abbildung
\[
D : K[x] \rightarrow K[x] \ \text{ gegeben durch } \ x^n \mapsto nx^{n-1}.
\]
Schr"anken wir diese Abbildung ein auf den endlichdimensionalen
invarianten(!) Unterraum $K[X]_{\leq d} = \bigoplus_{i = 0}^d Kx^d$
der Polynome vom Grad h"ochstens $d$, dann erhalten wir einen Endomorphismus
\[
\overline{D} \coloneqq D\mid_{K[x]_{\leq d}} : K[x]_{\leq d} \rightarrow K[x]_{\leq d}
\]
  
\begin{enumerate}
\item Wenn $K$ Charakteristik $0$ hat, also $\Q \subseteq K$,
  bestimmen Sie das Minimalpolynom von $\overline{D}$ und geben Sie
  eine Jordan-Basis an. % Machen Sie jeweils Ihren Rechenweg kennbar.
\item Wenn $K = \Z/p\Z$ der kleinste K"orper der Charakteristik $p >
  0$ ist, mit $p$ also prim, bestimmen Sie die Jordansche Normalform
  von $\overline{D}$.
\end{enumerate}

\end{aufgabe}

\begin{aufgabe}[4 Punkte]

Seien $F$ und $G$ kommutierende Endomorphismen eines Vektorraums $V$,
"uber dem K"orper $K$, und seien $F$ und $G$ beide nilpotent, zeigen
Sie dann, dass auch $F+G$ nilpotent ist.

Geben Sie zwei Endomorphismen an, die beide nilpotent sind, aber ihre
Summe nicht.
\end{aufgabe}


\begin{aufgabe}[4 Punkte]
Betrachte den $\R$-Vektorraum $V=C^0([0,1],\R)$ der stetigen,
reellwertigen Funktionen auf dem Intervall $[0,1]$.

Zeige, dass
$$
\langle f,g \rangle \coloneqq \int_0^1 f(x)g(x) dx, \quad f,g \in V,
$$
ein Skalarprodukt auf $V$ definiert.

\end{aufgabe}

\newpage
\bigskip
{\Large \bfseries Wiederholungsaufgaben}
\bigskip

\begin{aufgabe}
  Wir betrachten die linearen(!) Abbildungen 
  \begin{align*}
  F &\colon M(2 \times 2; \R) \to M(2 \times 2; \R) : X \mapsto AX, \ \text{mit} \ 
  A =
  \left( \begin{smallmatrix}a&b\\c&d\end{smallmatrix}\right) \ \text{fest}\\
  ev &\colon K[t]_{\le 4} \to \End(M(2 \times 2; \R)) : p \mapsto
  p(F)
  \end{align*}
  
  \begin{enumerate}
  \item
    W"ahlen Sie eine Basis $\mathcal{B}$ f"ur $M(2 \times 2; \R)$
    % die Basis $B =\left\{
    %   \left(
    %     \begin{smallmatrix}1&0\\0&0\end{smallmatrix}\right),
    %   \left(  \begin{smallmatrix}0&1\\0&0\end{smallmatrix}\right),
    %   \left(  \begin{smallmatrix}0&0\\1&0\end{smallmatrix}\right),
    %   \left(\begin{smallmatrix}0&0\\0&1\end{smallmatrix}
    %   \right)\right\}$
    eine Basis $\mathcal{P}$ f"ur $K[t]_{\leq 4}$ sowie eine Basis
    $\mathcal{B'}$ f"ur % die Basis $P = \{1, t, t^2, t^3, t^4\}$.  
    $\End(M(2 \times 2; \R))$.
    Bestimmen Sie die darstellenden Matrizen $M_B(F)$ und $M_P^{B'}(ev)$.
  \item
    Wollen Sie vielleicht doch andere Basen w"ahlen, um die
    darstellenden Matrizen zu berechnen?
  % Gleiche frage, aber jetzt ist $B =\left\{ \left( \begin{smallmatrix}1&0\\0&1\end{smallmatrix}\right),\left(  \begin{smallmatrix}1&1\\1&1\end{smallmatrix}\right),\left(  \begin{smallmatrix}1&0\\0&-1\end{smallmatrix}\right),\left(\begin{smallmatrix}1&1\\-1&1\end{smallmatrix} \right)\right\}$ und $P = \{1,1+ t, 1+t+t^2, 1+t+t^2+t^3, 1+t+t^2+t^3+t^4\}$. Die induzierte Basis $B'$ wird jetzt also durch die neue Basis $B$ induziert.
  \end{enumerate}
\end{aufgabe}

\begin{aufgabe}
Es seien $K$ ein K"orper, $V$ ein endlichdimensionaler $K$-Vektorraum und $F : V \rightarrow V$ ein $K$-Vektorraum Endomorphismus. Beweisen Sie die "Aquivalenz der folgenden Aussagen:


\begin{enumerate}
\item Es gibt $F$ invariante Unterr"aume $V_i \subset V \ i= 1,\dots , m$, so dass $V = V_1 \oplus \cdots \oplus V_m$
\item Es gibt eine Basis $B$ von $V$ so dass 
\[
M_B(F) = \begin{pmatrix}B_1 & 0 & \cdots& 0\\
0& \ddots & & \vdots \\
\vdots & & \ddots& 0 \\
0 & \cdots & 0 & B_m
\end{pmatrix}
\]
\end{enumerate}
\end{aufgabe}

\begin{aufgabe}
  "Ubersetzen Sie die folgende Aussage in die Sprache der
  Endomorphismen und der invarianten Unterr"aume.
  \begin{quotation}
    $F \in \End(V)$ hat eine darstellende Matrix, die eine obere
    Dreiecksmatrix mit Nullen auf der Diagonalen ist.
  \end{quotation}

  Benutzen Sie Ihre "Ubersetzung, um zu zeigen, dass dann $F^{\dim V}
  = 0$.
\end{aufgabe}

\begin{aufgabe}
Wir nehmen drei Polynome in $\Q[x]$
\begin{align*}
p &= x^8 + 3x^7 + 2x^6 + x^5 - x^4 - 3x^3 - 2x^2 - x\\
q &= 2x^7 + 2x^6 + x^4 - 2x^3 - 2x^2 - 1\\
r &= x^5 + x^4 - x^3 - 1,
\end{align*}
\begin{enumerate}
\item
  Bestimmen Sie ein Polynom $s$, so dass $(p,q) = (s)$
  und $a,b$, so dass $ap+bq=s$.
\item
  Bestimme ein Polynom $s$, so dass $(p,q,r) = (s)$.
 \end{enumerate}
\end{aufgabe}

\begin{aufgabe}

Es sei $R$ ein kommutativer Ring und $I, J \subset R$ Ideale. Dann definieren wir die Summe, das Produkt und der Durchschnitt von $I$ und $J$ wie
\begin{align*}
I + J &= \{ i + j \mid i \in I, j \in J \}\\
IJ &= \left\{ \sum_{k = 1}^n i_k j_k \mid i_k \in I, j_k \in J, k \geq 0 \right\}\\
I \cap J &= I \cap J.
\end{align*}
Zeigen Sie, dass diese drei Menge Ideale von $R$ sind. Zeigen Sie weiter dass $IJ \subseteq I \cap J$ und geben Sie ein Ring $R$ und Ideale $I, J \subset R$ f"ur die die umgegehrte Inklusion nicht gilt.  
\end{aufgabe}


\end{document}

%%% Local Variables: 
%%% mode: latex
%%% End: 
