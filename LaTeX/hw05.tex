\documentclass[a4,11pt]{article}

\usepackage{exercises}
\usepackage{macros}

\usepackage{graphicx}
\usepackage{ngerman}

\vltitel{Lineare Algebra 2}
\dozent{\small{Christian Haase}}
\assistent{\small{Jan Marten Sevenster}}
\tutoren{\small{%
    Theresa Graeber \\[-1ex] Eva Schinzel}}

\semester{Sommersemester 2023%
  % \raisebox{-10mm}[0pt][0pt]{%
  %   \parbox{0pt}{\includegraphics[width=27mm]{../../2015-ana1-L/Vorlesungsmaterial/ana1QR}}}
}


% https://math.ug/la2-ss21/sec-cayley-hamilton.html


\begin{document}
\vspace*{-17mm}
{
\kopf
}
% \vspace*{-5mm}
% \enlargethispage*{25mm}

\newcounter{chapter}
\ueblatt{5}{ Montag, 29.~Mai 2023 um 10h00 (ein PDF, jede Aufgabe auf neuer Seite beginnen)}

\begin{aufgabe}[4 Punkte]

  $D \colon$ Polynome ableiten. Kann man auf Polynome Grad $\le d$ einschr"anken
  
\begin{enumerate}
\item Wenn $\Q \subseteq K$, Minimalpolynom, Jordan-Basis
\item Wenn $p$ prim und $K = \Z/p\Z$, JNF
\end{enumerate}

\end{aufgabe}

\begin{aufgabe}[4 Punkte]

\end{aufgabe}


\begin{aufgabe}[4 Punkte]
Es sei $V$ ein $7$-dimensionaler $K$-Vektorraum und $\lambda \in K$. Finden Sie ein Endomorphismus $F$, so dass
\begin{align*}
\dim \ker (F -\lambda I_V)^1 &= 3\\
\dim \ker (F -\lambda I_V)^2 &= 6 \\
\dim \ker (F -\lambda I_V)^3 &= 7
\end{align*}.
Geben Sie eine Darstellende Matrix f"ur $F$ an, und begr"unden Sie wie Sie die gefunden haben.
\end{aufgabe}

Wiederholungsaufgaben

\begin{aufgabe}
  Ein paar Abbildungen ... Basen w"ahlen, darstellende Matrix bestimmen

  $F \colon M(2 \times 2; \R) \to M(2 \times 2; \R)$ $X \mapsto AX$
  for fixed $A =
  \left( \begin{smallmatrix}a&b\\c&d\end{smallmartix}\right)$  
  
  $ev \colon K[t]_{\le 4} \to \End(M(2 \times 2; \R))$ $p \mapsto
  p(F)$ ($F$ wie oben).
\end{aufgabe}

\begin{aufgabe}
  direkte Summe invarianter Unterr"aume iff Blockdiagonalform
\end{aufgabe}

\begin{aufgabe}
  zwei Polynome $p,q$ bestimme ein Polynom $s$, so dass $(p,q) = (s)$
  und $a,b$, so dass $ap+bq=s$

  drei Polynome $p,q,r$ bestimme ein Polynom $s$, so dass $(p,q,r) = (s)$
\end{aufgabe}

\begin{aufgabe}
Ideale $I,J$. Definiere/zeige: $I+J$, $I \cap J$, $IJ$ sind Ideale und
$IJ \subseteq I \cap J$. Beispiel f"ur $\neq$.  
\end{aufgabe}


\end{document}

%%% Local Variables: 
%%% mode: latex
%%% End: 
