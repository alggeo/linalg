\documentclass[a4,11pt]{article}

\usepackage{./exercises}
\usepackage{./macros}

\usepackage{graphicx}
\usepackage{ngerman}

\vltitel{Lineare Algebra 2}
\dozent{\small{Christian Haase}}
\assistent{\small{Jan Marten Sevenster}}
\tutoren{\small{%
    Theresa Graeber \\[-1ex] Eva Schinzel}}

\semester{Sommersemester 2023%
  % \raisebox{-10mm}[0pt][0pt]{%
  %   \parbox{0pt}{\includegraphics[width=27mm]{../../2015-ana1-L/Vorlesungsmaterial/ana1QR}}}
}
\begin{document}

\begin{aufgabe}[4 Punkte]
  Bestimmen Sie alle Eigenwerte der linearen Abbildung
  $$ \begin{array}{rccc}
       D \colon & C^\infty(\R,\R) &\to& C^\infty(\R,\R) \\
       & f &\mapsto& f' \ .
  \end{array} $$
  Dabei steht $C^\infty(\R,\R)$ f"ur den $\R$-Vektorraum der
  beliebig of differenzierbaren Funktionen $\R \to \R$.
\end{aufgabe}

\newpage

\begin{aufgabe}[4 Punkte]
  Geben Sie ein Beispiel f"ur einen K"orper $K$ und ein Polynom $f \in K[t]$ ohne
  Nullstellen in $K$, welches aber {\bfseries nicht} irreduzibel in $K[t]$ ist.
\end{aufgabe}

\newpage

\newpage

\begin{aufgabe}

Es sei $M \in \text{Mat}(9 \times 9; \mathbb{C})$ die Matrix
\[
M = 
\begin{pmatrix}
2 & 1 & 0 & 0 & 0 & 0 & 0 & 0 & 0\\
0 & 2 & 1 & 0 & 0 & 0 & 0 & 0 & 0\\
0 & 0 & 2 & 1 & 0 & 0 & 0 & 0 & 0\\
0 & 0 & 0 & 2 & 1 & 0 & 0 & 0 & 0\\
0 & 0 & 0 & 0 & 2 & 0 & 0 & 0 & 0\\
0 & 0 & 0 & 0 & 0 & 2 & 1 & 0 & 0\\
0 & 0 & 0 & 0 & 0 & 0 & 2 & 1 & 0\\
0 & 0 & 0 & 0 & 0 & 0 & 0 & 2 & 1\\
0 & 0 & 0 & 0 & 0 & 0 & 0 & 0 & 2\\

\end{pmatrix}
\]
Bestimmen Sie das Minimalpolynom.

\end{aufgabe}

\newpage

\begin{aufgabe}

Es sei $M \in \text{Mat}(6 \times 6; \mathbb{C})$ die Matrix
\[
M = 
\begin{pmatrix}
2 & 0 & 0 & 0 & 0 & 0 \\
0 & 1 & 0 & 0 & 0 & 0 \\
0 & 0 & 1 & 986756 & 0 & 0 \\
0 & 0 & 0 & 2 & 0 & 0 \\
0 & 0 & 0 & 0 & 2 & 0 \\
0 & 1 & 0 & 0 & 0 & 1 \\
\end{pmatrix}
\]
  Bringen Sie die in Jordansche Normalform und bestimmen Sie das Minimalpolynom.

  %alternativ: unm"ogliche Folge von Kernen von $(F - \lambda \id)^k$ \ldots
  %gibt Beispiel oder zeige, dass es nicht geht.
  
\end{aufgabe}

\newpage

\begin{aufgabe}

Es sei $M \in \text{Mat}(2 \times 2 ; \mathbb{C})$ die Matrix

\[
\begin{pmatrix}
5 & i\\
i & 2 \\
\end{pmatrix}.
\]
Entscheiden Sie, ob die Abbildung
\[
\mathbb{C}^2 \times \mathbb{C}^2 \rightarrow \mathbb{C} : (v, w) \rightarrow \transpose vM \overline{w} 
\]
ein Skalarprodukt auf $\mathbb{C}^2$ definiert und beweisen Sie das Ergebnis Ihrer "Uberlegung.

\end{aufgabe}

\newpage

\begin{aufgabe}
Wir betrachten den $\mathbb{Q}$-Vektorraum
\[
\mathbb{Q}\left[ \sqrt{-d}\right] = \left\{ p + q \sqrt{-d} \mid p,q \in \mathbb{Q} \right\} \subset \mathbb{C},
\]
wobei $d \in \mathbb{Z}_{>0}$ Quadratfrei ist. Auf diesen Vektorraum wird auf die folgende Weise eine Bilinearform $s$ definiert:
\[
s(v,w) = \frac{1}{4}\left((v + \overline{v})(w + \overline{w}) - d^{-1} (v - \overline{v})(w - \overline{w}) + 2\sqrt{-d}^{-1}(vw - \overline{vw})\right) \in \mathbb{Q}. 
\]
\begin{enumerate}
\item
W"ahlen Sie eine Basis $\mathcal{B}$ f"ur $\mathbb{Q}\left[ \sqrt{-d}\right]$ und geben Sie die darstellende Matrix $M_\mathcal{B}(s)$.
\item
Erweitern wir diese definition nach dem $\mathbb{R}$ Vektrorraum $\mathbb{R}\left[ \sqrt{-d}\right] \cong \mathbb{C}$, dann erhalten wir eine Bilinearform
\[
\overline{s} : \mathbb{R}\left[ \sqrt{-d}\right] \times \mathbb{R}\left[ \sqrt{-d}\right] \rightarrow \mathbb{R}
\]
Induziert diese Bilinearform einen Norm auf $\mathbb{R}\left[ \sqrt{-d}\right]$? (also ist die Abbildung $v \mapsto \sqrt{\overline{s}(v,v)}$ ein Norm?)
\end{enumerate}
\end{aufgabe}


\newpage

\begin{aufgabe}
Es sei $V$ ein endlichdimensionaler Vektorraum, der entwerder unit"ar oder euklidisch ist, und nehme $W \subseteq V$ einen Unterraum. Zeigen Sie, dass
\[
V = W \oplus W^\perp
\]
\end{aufgabe}

\newpage

\begin{aufgabe}
\begin{enumerate}
\item
Es sei
\[
A =
\begin{pmatrix}
-1 & -1 & -1\\
-1 & -1 & -1 \\
-1 & -1 & -2
\end{pmatrix}.
\]
Berechnen Sie alle Hauptminoren von $A$. Diese sind alle Determinanten der From
\[
\det \begin{pmatrix} a_{i_1 i_1} & \cdots & a_{i_1 i_k} \\ \vdots & \ddots & \vdots \\ a_{i_k i_1} & \cdots & a_{i_k i_k} \\ \end{pmatrix}, \emptyset \neq \{ i_1  < \cdots < i_k \} \subseteq \{ 1, 2, 3\} .
\]
\item
Leiten Sie notwendige und hinreichende Bedingungen ab f"ur die Hauptminoren einer Matrix $M \in \text{Mat}(n \times n ; \mathbb{R})$, die beschreiben dass $M$ negativ semi-definit ist.
\end{enumerate}
\end{aufgabe}

\newpage

\begin{aufgabe}
Erinneren Sie sich an den folgenden "aquivalente Definitionen die wir bisher gefunden haben f"ur positiv semidefinite Matrizen $M \in \text{Mat}(n\times n ; \mathbb{C})$ die selbstadjungiert sind
\begin{enumerate}
\item Alle Eigenwerte sind nichtnegativ
\item  $\transpose v M \overline{v} \geq 0 \ \forall v \in \mathbb{C}^n$
\item Es existiert $W \in \text{Mat}(n\times n ; \mathbb{C})$ so dass $M = W W^* = W \transpose \  \overline{W}$
\end{enumerate}
Diese Bedingung ist zu den obigen "aquivalent.
\begin{equation}
\label{one}
\text{Spur}(MP) \geq 0 \ \forall P \in  \text{Mat}(n\times n ; \mathbb{C}) \text{ positiv semidefinit}.
\end{equation}

Beweisen Sie eine der beiden Implikationen. Also \ref{one} impliziert positiv semidefinit oder positiv semidefinit impliziert \ref{one}.


\end{aufgabe}

\end{document}

%%% Local Variables:
%%% mode: latex
%%% End:
