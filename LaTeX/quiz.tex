\documentclass[a4]{report}
 
\usepackage{bbm}
\usepackage[german]{../../me-exercises}
\usepackage{../../entdecken_macros}
\graphicspath{{../../texmf/Bilder/}}
\usepackage{../klaus}
% \usepackage{tabularx}
% \usepackage{multirow}
\usepackage{tikz}
\usepackage{graphicx}
\usepackage{ngerman}
\usepackage{hyperref}
\usepackage{multicol}

\vltitel{Mathematik entdecken 1}
%\dozent{Christian Haase \\ Anina Mischau}
%\assistent{Lena Walter}
%\tutoren{Martin G"unther, Marie Kristin Piel \\ Marvin Rominger, Hannah Zabel}
%
\semester{Wintersemester 2019/20}

\begin{document}
{
\kopf
}
\quiz{2}{}

\begin{center}
  \begin{tabular}{llllll}
    \tutorienfrage{Elisabeth} \qquad
    \tutorienfrage{Kristin} \qquad
    \tutorienfrage{Taha} \qquad
  \end{tabular}
\end{center}

\musterlsgfrage

\begin{aufgabe}[4 Punkte]
  Suchen Sie sich eine der beiden folgenden Implikationen
  aus. Entscheiden Sie, ob sie f"ur alle ganzen Zahlen $a,b,c$ wahr
  ist und begr"unden Sie bzw.\ geben Sie ein Gegenbeispiel.
  $$ \left( a \text{ teilt } bc \right) \ \wennSo \ \left(\left( a
      \text{ teilt } b \right) \oder  \left(a \text{ teilt } c
    \right)\right) $$ 
  $$ \left( a \text{ teilt } bc \right) \ \leftarrow \ \left(\left( a
      \text{ teilt } b \right) \oder  \left(a \text{ teilt } c
    \right)\right) $$ 
\end{aufgabe}

\end{document}

%%% Local Variables:
%%% mode: latex
%%% End:
