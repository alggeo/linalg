\documentclass[a4,11pt]{article}

\usepackage{./exercises}
\usepackage{./macros}

\usepackage{graphicx}
\usepackage{ngerman}

\vltitel{Lineare Algebra 2}
\dozent{\small{Christian Haase}}
\assistent{\small{Jan Marten Sevenster}}
\tutoren{\small{%
    Theresa Graeber \\[-1ex] Eva Schinzel}}

\semester{Sommersemester 2023%
  % \raisebox{-10mm}[0pt][0pt]{%
  %   \parbox{0pt}{\includegraphics[width=27mm]{../../2015-ana1-L/Vorlesungsmaterial/ana1QR}}}
}
\begin{document}

\begin{aufgabe}[4 Punkte]
  Bestimmen Sie alle Eigenwerte der linearen Abbildung
  $$ \begin{array}{rccc}
       D \colon & C^\infty(\R,\R) &\to& C^\infty(\R,\R) \\
       & f &\mapsto& f' \ .
  \end{array} $$
  Dabei steht $C^\infty(\R,\R)$ f"ur den $\R$-Vektorraum der
  beliebig of differenzierbaren Funktionen $\R \to \R$.
\end{aufgabe}

\newpage

\begin{aufgabe}[4 Punkte]
  Geben Sie ein Beispiel f"ur einen K"orper $K$ und ein Polynom $f \in K[t]$ ohne
  Nullstellen in $K$, welches aber {\bfseries nicht} irreduzibel in $K[t]$ ist.
\end{aufgabe}

\newpage

\newpage

\begin{aufgabe}

Es sei $M \in \text{Mat}(9 \times 9; \mathbb{C})$ die Matrix
\[
M = 
\begin{pmatrix}
2 & 1 & 0 & 0 & 0 & 0 & 0 & 0 & 0\\
0 & 2 & 1 & 0 & 0 & 0 & 0 & 0 & 0\\
0 & 0 & 2 & 1 & 0 & 0 & 0 & 0 & 0\\
0 & 0 & 0 & 2 & 1 & 0 & 0 & 0 & 0\\
0 & 0 & 0 & 0 & 2 & 0 & 0 & 0 & 0\\
0 & 0 & 0 & 0 & 0 & 2 & 1 & 0 & 0\\
0 & 0 & 0 & 0 & 0 & 0 & 2 & 1 & 0\\
0 & 0 & 0 & 0 & 0 & 0 & 0 & 2 & 1\\
0 & 0 & 0 & 0 & 0 & 0 & 0 & 0 & 2\\

\end{pmatrix}
\]
Bestimmen Sie das Minimalpolynom.

\end{aufgabe}

\newpage

\begin{aufgabe}

Es sei $M \in \text{Mat}(6 \times 6; \mathbb{C})$ die Matrix
\[
M = 
\begin{pmatrix}
2 & 0 & 0 & 0 & 0 & 0 \\
0 & 1 & 0 & 0 & 0 & 0 \\
0 & 0 & 1 & 986756 & 0 & 0 \\
0 & 0 & 0 & 2 & 0 & 0 \\
0 & 0 & 0 & 0 & 2 & 0 \\
0 & 1 & 0 & 0 & 0 & 1 \\
\end{pmatrix}
\]
  Bringen Sie die in Jordansche Normalform und bestimmen Sie das Minimalpolynom.

  %alternativ: unm"ogliche Folge von Kernen von $(F - \lambda \id)^k$ \ldots
  %gibt Beispiel oder zeige, dass es nicht geht.
  
\end{aufgabe}


\begin{aufgabe}

Es sei $M \in \text{Mat}(2 \times 2 ; \mathbb{C})$ die Matrix

\[
\begin{pmatrix}
1 & i + 1\\
i - 1 & i \\
\end{pmatrix}.
\]
Entscheiden Sie, ob die Abbildung
\[
\mathbb{C}^2 \times \mathbb{C}^2 \rightarrow \mathbb{C} : (v, w) \rightarrow v^\dagger (v^* , \overline{v^T} ? ) M w 
\]
ein Skalarprodukt auf $\mathbb{C}^2$ definiert und beweisen Sie das Ergebnis Ihrer "Uberlegung.

\end{aufgabe}

\end{document}

%%% Local Variables:
%%% mode: latex
%%% End:
