\documentclass[a4,11pt]{article}

\usepackage{exercises}
\usepackage{macros}

\usepackage{graphicx}
\usepackage{ngerman}

\vltitel{Lineare Algebra 2}
\dozent{\small{Christian Haase}}
\assistent{\small{Jan Marten Sevenster}}
\tutoren{\small{%
    Theresa Graeber \\[-1ex] Eva Schinzel}}

\semester{Sommersemester 2023%
  % \raisebox{-10mm}[0pt][0pt]{%
  %   \parbox{0pt}{\includegraphics[width=27mm]{../../2015-ana1-L/Vorlesungsmaterial/ana1QR}}}
}

\begin{document}
\vspace*{-17mm}
{
\kopf
}
% \vspace*{-5mm}
\enlargethispage*{25mm}

\begin{center}
  \LARGE \bfseries Inhaltliches \makebox[0pt]{\hspace*{120mm}\tiny Stand: \today}
\end{center}

\bigskip

{\bfseries Lernziele.}
Die Studierenden \ldots
\begin{compactitem}
%  Die Studentinnen und Studenten beherrschen algebraische
%  Schlussweisen und Beweisstrategien und sind mit der Problematik von
%  Normalformen für lineare Abbildungen und quadratische Formen
%  vertraut. Sie können zu passenden Problemstellungen auch aus
%  anderen Wissenschaften geeignete mathematische Formulierungen
%  entwickeln, dazu die richtigen Lösungen finden und ihre Ergebnisse
%  wieder im Anwendungskontext richtig interpretieren.
  
% \item Die Studentinnen und Studenten k"onnen vom Zufall abh"angige Ph"anomene
% modellieren und in diesen Modellen argumentieren. Sie k"onnen Daten mit
% elementaren Mitteln der beschreibenden Statistik ordnen, darstellen
% und durch Kenndaten charakterisieren. Die Studentinnen und Studenten
% verf"ugen "uber einen Fundus an Beispielen, die die Fallstricke der
% schlie"senden Statistik illustrieren. Sie k"onnen so den Gebrauch von
% Statistiken in der "offentlichen Diskussion kritisch bewerten. Sie
% begreifen die Stochastik aus ihrer (Ideen-)Geschichte und ihren
% Anwendungen heraus als lebendige Wissenschaft, und sie k"onnen vor
% diesem Hintergrund den Schulbezug der vermittelten Inhalte selbstst"andig
% herstellen.
  
% \item Die Studierenden erkennen in der Tagespresse das Auftreten
%   bedingter Wahrscheinlichkeiten und k"onnen die betreffende Situation
%   pr"azise mathematisch modellieren.
% \item \ldots beherrschen algebraische Schlussweisen und
%   Beweisstrategien. Sie erkennen 
%   Inhaltsliste) aus der
%   Tagespresse mathematisch einordnen und auf Plausibilit"at und
%   versteckte Annahmen pr"ufen.
\item \ldots k"onnen mithilfe von Resultaten aus der
  Vorlesung (siehe Inhaltsliste) einfache algebraische Aussagen
  eigenst"andig beweisen.
\item \ldots k"onnen eigenst"andig Beispiele und Nicht-Beispiele von
  Strukturen mit vorgegebenen Eigenschaften konstruieren.
\item \ldots verf"ugen "uber "Uberblickswissen und k"onnen
  eigenst"andig die wichtigsten S"atze, Definitionen und Beispiele der
  Vorlesung in logische/argumentative Beziehung setzen.
\end{compactitem}
\bigskip

{\bfseries Inhalte.}
\begin{compactitem}
\item Eigenwerte und Eigenvektoren, Diagonalisierbarkeit,
  Satz von Cayley-Hamilton, Normalformen
\item Vektorr"aume mit Skalarprodukt, Orthogonalit"at, Isometrien,
  (selbst-)adjungierte Abbildungen, Orthonormalisierung
\item ausgew"ahlte Anwendungen
\end{compactitem}

\mbox{}
\vfill

\begin{center}
  \LARGE \bfseries Organisatorisches
\end{center}
%\bigskip

%Willkommen zur Vorlesung \emph{Mathematik entdecken} im
%Wintersemester 2017/18. Sie richtet sich an Studierende der
%Lehramtsstudieng"ange.
%Inhaltlich besch"aftigen wir uns mit grundlegenden algebraischen
%Strukturen (Gruppen, Ringe, K"orper) und deren Anwendungen in
%Zahlentheorie und Geometrie.
%
%Weitere Schwerpunkte der Veranstaltung sind das Erlernen
%mathematischer Denk- und Arbeitsweisen, sowie die Bef"ahigung, Bez"uge
%universit"arer Inhalte und Sichtweisen zur Schulmathematik zu erkennen
%und zu nutzen.
\bigskip

{\bfseries Termine.} % \medskip
%
% Team-Meeting: ???????
\newline
\begin{tabular}[t]{llr@{--}ll}
  Vorlesung 
  & Montag      & 12&14 & H"orsaal % 2 in der Rostlaube\footnotemark \\
  001 in der Arnimallee 3  \\
  % \addtocounter{footnote}{-1}%
  & Mittwoch & 12&14 & H"orsaal 001 in der Arnimallee 3 \\
  Zentral-"Ubung  
  & Montag      &  14&16 & H"orsaal 001 in der Arnimallee 3 \\
  Tutorien
  & Dienstag & 08&10 & Seminarraum 032 in der A6 \\ % Eva
  & Mittwoch & 10&12 & Seminarraum 024 in der A3 \\ % Theresa
  & Mittwoch & 14&16 & Seminarraum 032 in der A6 \\ % Eva
  & Donnerstag & 08&10 & Seminarraum 024 in der A3 \\ % Theresa
%  & Donnerstag & 16&18 & Seminarraum 119 in der A3 \\ % Theresa??
 %
  %Probeklausur 
  %& Montag, 06.06. & 10&12 & H"orsaal Hs2/Rostlaube \\
  Abschlussklausur
  & Freitag, 28.07.  & 10&12 & TBA \\ % H"ors"ale A,C Henry-Ford-Bau\\
  Nachklausur
  & Donnerstag, 12.10. & 10&12 & TBA \\ % H"ors"ale A,C Henry-Ford-Bau  \\
\end{tabular}
% \footnotetext{Ausnahmen beachten: }
\bigskip

% {\bfseries Zielgruppe.}
% Die Vorlesung richtet sich in erster Linie an

{\bfseries "Ubungen.}
\begin{itemize}
\item \textbf{Zentral-"Ubung.} Es wird eine w"ochentliche, zentrale
  "Ubung geben. Hier wird zu Beginn ein kurzes Quiz geschrieben,
  welches dann korrigiert und bewertet wird. Im Anschluss werden
  ausgew"ahlte Inhalte der Vorlesung exemplarisch aufgearbeitet. Die
  Teilnahme an dieser Veranstaltung ist verpflichtend f"ur die
  Best"atigung der Studienleistung "`regelm"a"sige Teilnahme"'. 
\item \textbf{Tutorien.}
Neben dieser zentralen Veranstaltung werden w"ochentliche Tutorien
angeboten, in welchen Sie Hilfe beim Bearbeiten der Hausaufgaben
bekommen und ggf. Pr"asenzaufgaben bearbeiten.
Die Tutorien dienen dazu, den Stoff der Vorlesung vor- und
nachzubereiten. Es soll insbesondere auch ein Ort sein, wo
Sie all Ihre Fragen % und Ihre bl"oden Fragen loswerden.
stellen.

Die Tutorien sind explizit {\em nicht} der Ort, wo Ihnen fertige
L"osungen pr"asentiert werden. 
% \item \textbf{Tutorien'.}
% Neben dieser zentralen Veranstaltung werden verschiedene Tutorien
% angeboten: sie k"onnen Vorlesungsinhalte (auch aus der Linearen
% Algebra~1) wiederholen, Hilfe beim Herangehen oder beim Aufschreiben
% der Hausaufgaben bekommen oder weiterf"uhrende Ideen diskutieren.
% Welches Tutorium welchen Schwerpunkt hat, wird alle drei Wochen
% aufgrund einer Whiteboard-Abstimmung neu entschieden. Sie k"onnen auch
% zu mehreren Tutorien gehen.
% 
% Die Tutorien sind aber explizit {\em nicht} der Ort, wo Ihnen fertige
% L"osungen pr"asentiert werden. 
\end{itemize}
% Wir stellen Ihnen gescannte L"osungen 
% zur Verf"ugung.
% Hier ist Ihre Mitarbeit gefragt. Kommen Sie vorbereitet! Bringen Sie
% Fragen mit!!

%\smallskip
%
%W"ahrend der Tutorien werden Sie randomisiert vom Tutor aufgefordert,
%L"osungen, f"ur die Sie Punkte erhalten haben, zu
%erkl"aren. Gegebenenfalls erhalten Sie die Gelegenheit, Ihre L"osungen
%Lena Walter oder Christian Haase zu erkl"aren.
%An dieser Stelle sei darauf hingewiesen, dass in den Tutorien die
%Pflicht zur regelm"a"sigen Teilnahme besteht.
\bigskip

{\bfseries Aktive Teilnahme.} 
\newline
% Der Autor nimmt seine Leser als intelligente und interessierte Menschen an,
% die ein im Vergleich zur Schule deutlich erweitertes Wissen und vertieftes Ver-
% ständnis der Mathematik erwerben möchten. Über ein zeilenweise ausgeführtes
% Nach-Denken hinaus wird ein aktives Mit- und Weiterdenken ebenso erwartet
% wie die Bereitschaft, die formal-sprachlichen und systematischen Aspekte der
% Mathematik zu erlernen. Nach Überzeugung des Autors findet vieles, was selbst
% gedacht werden kann, tatsächlich besser im eigenen Kopf als auf dem bedruckten
% Papier statt, und die Qualität der Anregungen scheint ihm dabei weitaus wichti-
% ger zu sein als die Quantität der Hilfestellungen. Alles zu kommentieren und im
% Detail auszuführen, hieße, das Kind, das ja den Windeln längst entstiegen ist, auf
% den Berg hochzutragen. Das Plakat „Beispiele und Musterlösungen“ ist oft nur
% Ausdruck des (menschlichen) Wunsches, die zweifellos anstrengende Geistestä-
% tigkeit, mathematische Definitionen, Sätze und Beweise zu durchdringen, doch
% irgendwie vermeiden zu können. Die Mathematik ist aber kein einfacher Auto-
% mat, den man sich vielleicht tatsächlich besser durch zwei oder drei beispielhafte
% Bedienungen erklären lässt, anstatt seine Betriebsanleitung zu studieren. In der
% Mathematik sind Text und Objekt oft nur schwer zu unterscheiden, und im Zen-
% trum der mathematischen Texte stehen seit den alten Griechen, bei denen die
% Mathematik als Wissenschaft beginnt, Definitionen, Sätze und Beweise. Sie ent-
% halten sehr viel hochwertiges Gedankenfutter, sehr viele beispielhafte Lösungs-
% muster. Nur durch ihre Durchdringung und Aneignung wird das, was zu Beginn
% schwerfällt, irgendwann einmal einfach und leicht − wie etwa der Supremums-
% oder der Grenzwertbegriff −, und nur durch sie wird das, was zu Beginn unpro-
% blematisch erscheint, irgendwann einmal in seiner ungeahnten Komplexität er-
% kennbar − wie etwa der Begriff der reellen Zahl.
Montags wird auf der
\href{https://mycampus.imp.fu-berlin.de/x/wPdhCU}{Whiteboard Kurs-Seite} 
ein "Ubungsblatt mit drei
Aufgaben unterschiedlicher Schwierigkeit ausgegeben.
Ihre L"osungen sollen Sie {in Einzelabgabe} % im Tutorium 
bis Montag 10:00 der darauf folgenden Woche im Whiteboard hochladen. Das Quiz
orientiert sich "ublicherweise inhaltlich an einer der "Ubungsaufgaben.
Die beiden anderen Aufgaben werden korrigiert und bewertet.\footnote{%
Um den entstehenden Korrekturaufwand ertr"aglich zu halten behalten wir uns vor,
aus den verbleibenden zwei Aufgaben eine zu korrigierende auszuw"ahlen.}
% des "Ubungszettels nach dem
% Abgabetermin im Zufallsverfahren eine Aufgabe ausgew"ahlt.
% Diese Aufgabe wird korrigiert und bewertet.

Im Verlauf des Semesters ist ein Schreiblabor geplant. Einzelheiten
zum "Ubungsbetrieb w"ahrend der Laborwochen werden noch bekannt gegeben.

% W"ahrend des Semesters werden Sie in Kleingruppen ein Studienprojekt
% bearbeiten. Details hierzu werden in der zweiten Semesterwoche bekannt
% gegeben.



% Ich denke an sowas wie kein Quiz aber pro Woche zwei Aufgaben, die
% solange korrigiert werden bis sie vern"unftig aufgeschrieben sind. Das
% w"aren bei zwei mal zwei Laborwochen 8 Aufgaben von denen 6 richtig
% sein m"ussen.(?)

% {\bfseries Aktive Teilnahme.} 
% > Quoting Lutz Heindorf <heindorf@zedat.fu-berlin.de>:
% >> Hallo Christian,
% >> Dein Tutor betreut zwei Gruppen (jeweils eine Veranstaltung) und
% >> korrigiert die zugehörigen Hausaufgaben. Allerdings wird am 
% >> Fachbereich in Zweiergruppen abgegeben.
% >> Anfänglich sollte der Tutor also in der Grössenordnung 60 Studenten 
% >> und 30 Zettel haben.
\smallskip

Um die Studienleistung "`aktive Teilnahme"' am Ende der Veranstaltung
best"atigt zu bekommen, m"ussen die folgenden Kriterien erf"ullt sein:
\begin{itemize}
\itemsep0pt
\item Erreichen von mindestens {\em 50\%} der maximal
erreichbaren Hausaufgabenpunkte, die zwei Hausaufgabenzettel mit 
den niedrigsten erreichten Punktzahlen gehen dabei nicht mit in die
Wertung ein,
\item Erreichen von mindestens {\em 40\%} der maximal
erreichbaren Quizpunkte, die zwei Quiz mit 
den niedrigsten erreichten Punktzahlen gehen dabei nicht mit in die
Wertung ein,
\item erfolgreiche Teilnahme am Schreiblabor.
% \item erfolgreiche Bearbeitung und Pr"asentation des Studienprojektes.
\end{itemize}
%Es wird dringend empfohlen, eine Punktzahl "uber {\em 60\%}
%anzustreben.
% sowie im Tutorium mindestens eine gute Frage stellen.
\bigskip

%{\bfseries Doppelte Korrektur.} % in den ersten Wochen.}
%F"ur wenige ausgew"ahlte "Ubungsaufgaben bieten wir Ihnen eine doppelte
%Korrektur an. Nach der (ersten) Abgabe einer "Ubung werden wir Ihre
%Aufschriebe im Hinblick auf die schriftliche Umsetzung Ihrer L"osungen
%kommentieren und auf sprachliche Ungenauigkeiten oder Zweideutigkeiten
%hinweisen. Diese Vorkorrektur wird im Tutorium zur"uckgegeben; sie
%wird nicht bewertet. Sie erhalten dann die Gelegenheit Ihre L"osungen
%zu "uberarbeiten und zusammen mit dem folgenden "Ubungsblatt erneut
%abzugeben. Zun"achst gar nicht oder nur rudiment"ar bearbeitete
%Aufgaben werden bei der zweiten Korrektur nicht gewertet. Diesen
%aufwendigeren Korrekturmodus k"onnen wir aber nicht das ganze Semester
%"uber leisten.
%
%\medskip

% {\bfseries Literatur.}
% Die Vorlesung richtet sich in weiten Teilen nach \textcolor[rgb]{1,0,0}{???}
%
%\smallskip
%
%\quad

\end{document}


%%% Local Variables:
%%% mode: latex
%%% End:
