\documentclass[a4,11pt]{article}

\usepackage{./exercises}
\usepackage{./macros}
\usepackage{mathtools}
\usepackage{tikz-cd}

\usepackage{graphicx}
\usepackage{ngerman}

\vltitel{Lineare Algebra 2}
\dozent{\small{Christian Haase}}
\assistent{\small{Jan Marten Sevenster}}
\tutoren{\small{%
    Theresa Graeber \\[-1ex] Eva Schinzel}}

\semester{Sommersemester 2023%
  % \raisebox{-10mm}[0pt][0pt]{%
  %   \parbox{0pt}{\includegraphics[width=27mm]{../../2015-ana1-L/Vorlesungsmaterial/ana1QR}}}
}

\DeclareMathOperator{\End}{End}
\DeclareMathOperator{\Span}{span}
% \DeclareMathOperator{\ker}{ker}
\DeclareMathOperator{\im}{im}
\newcommand{\bonusitem}{\item\hspace*{-2.4mm}*\ }


\begin{document}
\vspace*{-17mm}
{
\kopf
}
% \vspace*{-5mm}
% \enlargethispage*{25mm}

\newcounter{chapter}
\ueblatt{12}{ Montag, 17.~Juli 2023 um 10h00}


\begin{aufgabe}[4 Punkte]
lineare Abb $V \to W$ dim $n$ resp. $m$, Rang $r$ kann man Basen
w"ahlen mit Einheitsmatrix $E_r$ und sonst Nullen

Warum folgt nicht sofort der SVD-Satz?

Kann man umgekehrt den obigen Satz einfach aus dem SVD-Satz folgern?  
\end{aufgabe}

\begin{aufgabe}[4 Punkte]
  Bsp Internet aus Fig. 2.1

  verifiziere, dass Eig$(A;1)$ eindimensional ist.

  Wir wollen den Pagerank manipulieren, ohne mit den Webmastern f"ur
  die Seiten $1$ bis $4$ zu reden. Dazu f"ugen wir ein Netz von fake
  Seiten ein, die sich untereinander verlinken und von denen einige
  auf die von uns pr"aferierte Seite $2$ verweisen.

  Wie "andert sich der Pagerank? (Beispiel oder Aussage mit Beweis).

  Durch Zufall lernen wir Webmaster $4$ auf der Fachbereichsparty
  kennen. Wir k"onnen sie "uberzeugen, einen Link auf eine unserer
  fake Seiten auf ihre Seite zu stellen. Wie "andert sich der Pagerank?
  
  
\end{aufgabe}


\begin{aufgabe}[4 Punkte]
Ein Spaziergang der L"ange $\ell \in \N$ in einem Internet $G=(V,E)$ ist
eine Folge von Seiten $v_0, \ldots, v_\ell \in V$ (Wiederholungen
erlaubt), so dass f"ur $i=1, \ldots, \ell$ $(v_{i-1},v_i) \in E$.

Zu einer Matrix $A \in M(n \times n; \R)$ mit nicht-negativen
Eintr"agen k"onnen wir ein Internet auf den Seiten $V = \{ 1, \ldots,
n \}$ definieren: $(i,j) \in E :\Leftrightarrow a_{ij} > 0$.

Zeigen Sie, dass f"ur $i,j \in V$ der $ij$-Eintrag der $\ell$-ten
Potenz $A^\ell$ genau dann positiv ist, wenn es in $G=(V,E)$ einen
Spaziergang der L"ange $\ell$ von $j$ nach $i$ gibt.
  
\end{aufgabe}

\end{document}

%%% Local Variables: 
%%% mode: latex
%%% End: 
