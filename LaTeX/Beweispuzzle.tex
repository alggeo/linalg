\documentclass[a4paper%%%,12pt
]{report}

\usepackage[german]{./exercises}
\usepackage{./macros}
\DeclareMathOperator{\Span}{span}


\begin{document}
\Large

{\bfseries Satz:} Sei $F$ ein unit"arer Endomorphismus des
endlichdimensionalen unit"aren Vektorraums $V$. Dann besitzt $V$ eine
Orthonormalbasis aus Eigenvektoren von $F$. Insbesondere ist $F$
diagonalisierbar.
\medskip

{\bfseries Beweis.}

\begin{itemize}[\mbox{}]
\addtolength{\itemsep}{4mm}
% \begin{enumerate}
\item Wir f"uhren Induktion "uber $n:=\dim V$. 
\\ F"ur $n = 1$ ist nichts zu beweisen.
\item Sei also $n \ge 2$ und $P_F = (\lambda_1 - t) \cdot \ldots \cdot
  (\lambda_n - t)$ mit $\lambda_1, \ldots, \lambda_n \in \C$
  die nach dem Fundamentalsatz der Algebra existierende
  Linearfaktorzerlegung des charakteristischen Polynoms von $F$.
\item Zum Eigenwert $\lambda_1$ w"ahlen wir einen Eigenvektor $v_1$
  mit $\|v_1\| = 1$. 
\item Wir betrachten das orthogonale Komplement zur Geraden
  $\Span(v_1)$: $W:= \Span(v_1)^\perp$.
\quad Wir wissen, dass $\dim W = n-1$.
\\ Wir zeigen nun, dass $W$ invariant unter $F$ ist:
\item Sei $w \in W$. % also $\langle v_1, w \rangle = 0$.
   Dann ist $$\lambda_1 \langle v_1, F(w) \rangle
  = \langle \lambda_1 v_1, F(w) \rangle = \langle F(v_1), F(w) \rangle
  = \langle v_1, w \rangle = 0 \,.$$
\item Wegen $|\lambda_1|=1$ ist $\lambda_1 \neq 0$, sodass $\langle
  v_1, F(w) \rangle = 0$ folgt.
\item Damit ist $F(w) \in W$ und die Invarianz gezeigt.
\item Die Einschr"ankung $F|_W$ von $F$ auf $W$ ist immer noch
  unit"ar.
\item Wir k"onnen also die Induktionsannahme auf $F|_W$
  anwenden:
\item Es gibt eine Orthonormalbasis $v_2, \ldots, v_n$ von $W$, die
  aus Eigenvektoren von $F|_W$ besteht.
\\ Diese Vektoren sind dann offenbar auch Eigenvektoren von $F$.
\item Damit ist $v_1, \ldots, v_n$ eine Basis von $V$, sie ist orthonormal
  und besteht aus Eigenvektoren von $F$.
\item \hfill $\Box$
\end{itemize}
% \end{enumerate}
  
\newpage

\begin{itemize}[\mbox{}]
%%% \addtolength{\itemsep}{5mm}
\item {\bfseries Definition:} Eine \emph{Linie} ist eine Teilmenge
  $\ell$ der Ebene $\R^2$, sodass es zwei Punkte $P$ und $Q$ im $\R^2$
  gibt, wobei $Q$ nicht der Ursprung $(0,0)$ ist, und $\ell$ gleich
  der Menge aller Punkte der Form $P+t Q$ f"ur reelle Zahlen $t$ ist.
  \medskip

  Eine \emph{Gerade} ist eine Teilmenge $g$ der Ebene $\R^2$, sodass
  es reelle Zahlen $a,b,c$ gibt, wobei $a$ und $b$ nicht beide $0$
  sind, und $g$ gleich der Menge aller Punkte $(x,y)$ im $\R^2$ ist,
  die die Gleichung $ax+by=c$ erf"ullen.
\item {\bfseries Satz:} Jede Gerade ist eine Linie.
\item {\bfseries Beweis.}
% \item Es sei $P=(p_{1},p_{2})\in\R^2,\, Q=(q_{1},q_{2})\in\R^2\setminus\left\{(0,0)\right\}$ und $\ell=\left\{P+t Q \suchthat t \in \R \right\} $ eine Linie.
% \item Wir setzen $a:=-q_{2}, b:=q_{1}$ und $c:=-q_{2}p_{1}+q_{1}p_{2}$.
% \item Weil $q_{1}$ und $q_{2}$ nicht beide 0 sind, sind auch $a$ und $b$ nicht beide 0.
% \item Daher ist $g:=\{(x,y)\in\R^2\suchthat ax+by=c\}$ eine Gerade.
% \item Man rechnet nach, dass gilt $g=\ell$.
% \item Daher ist jede Linie eine Gerade.
\item Seien $a,b,c$ reelle Zahlen, sodass $a$ und $b$ nicht beide $0$
  sind, und sei $g \subset \R^2$ die Menge aller Punkte $(x,y)$ im
  $\R^2$, die die Gleichung $ax+by=c$ erf"ullen.
\item Wir nehmen an, dass $a$ nicht $0$ ist. (Der Beweis funktioniert
  analog, wenn $b$ nicht $0$ ist.)
\item Wir setzen $P=(p_{1},p_{2}):=(c/a,0)$

  und $Q=(q_{1},q_{2}):=(b,-a)$.
\item Weil $a$ nicht $0$ ist, ist auch $Q$ nicht der Ursprung.
\item Betrachte die Menge $\ell \subset \R^2$ aller Punkte der Form
  $P+tQ$ f"ur reelle Zahlen $t$.
\item Wir zeigen $g=\ell$:
\item Wenn $(x,y)$ ein Punkt in $\ell$ ist, gibt es eine reelle Zahl
  $t$, sodass $(x,y)$ gerade der Punkt $P+tQ$ ist.
\item Man rechnet nach, dass die Gleichung $ax+by=c$ erf"ullt ist,
  also ist $(x,y)$ ein Punkt in $g$.
\item Sei umgekehrt $(x,y)$ ein Punkt in $g$.
\item Falls $q_1$ nicht $0$ ist, w"ahle f"ur $t$ die Zahl
  $\displaystyle \frac{x-p_1}{q_1}$.
  
  Falls $q_2$ nicht $0$ ist, w"ahle f"ur $t$ die Zahl $\displaystyle
  \frac{y-p_2}{q_2}$.
\item Unter Verwendung von $ax+by=c$ und der Definitionen von $P$ und
  $Q$ rechnet man nach, dass $(x,y)$ gerade der Punkt $P +tQ$ ist,
  also ist $(x,y)$ ein Punkt in $\ell$.\hfill$\Box$
\end{itemize}

%%% \bigskip

%%% \bigskip
\newpage

\begin{itemize}[\mbox{}]
%%% \addtolength{\itemsep}{5mm}
\item {\bfseries Definition:} $\ell\subseteq\R^2$ ist eine Linie
  $:\Leftrightarrow$

  $\exists P\in\R^2 \colon \exists
  Q\in\R^2\setminus\left\{(0,0)\right\} \colon \ell=\left\{P+t Q
    \suchthat t \in \R \right\}$.
  \bigskip
  
  $g\subseteq\R^2$ ist eine Gerade $:\Leftrightarrow$

  \mbox{%
    $\exists a,b,c\in\R\colon (a,b)\neq (0,0) \ \wedge \ 
    g=\left\{(x,y) \in \R^2 \suchthat ax+by=c\right\}$.
  }
\item {\bfseries Satz:} $\ell \subseteq \R^2$ ist eine Gerade $\Leftarrow
  \ell \subseteq \R^2$ ist eine Linie.
\item {\bfseries Beweis.}
\item Sei $P=(p_{1},p_{2})\in\R^2,\,
  Q=(q_{1},q_{2})\in\R^2\setminus\left\{(0,0)\right\}$ und
  $\ell=\left\{P+t Q \suchthat t \in \R \right\}$.
\item $a:=-q_{2}, b:=q_{1}$ und $c:=-q_{2}p_{1}+q_{1}p_{2}$.
\item $(q_{1},q_{2})\neq (0,0)\Rightarrow (a,b)\neq (0,0)$.
\item $\Rightarrow g:=\{(x,y)\in\R^2\suchthat ax+by=c\}$ ist eine
  Gerade.
\item Wir zeigen $g=\ell$:
\item $(x,y) \in \ell$ $\Rightarrow$ $\exists t \in \R \colon (x,y) =
  P+tQ$.
\item Man rechnet nach, dass $ax+by=c$.

  $\Rightarrow$ $(x,y) \in g$.
\item Sei umgekehrt $(x,y) \in g$.
\item % \mbox{%
    % Falls $q_1 \neq 0$, setze $t := \displaystyle \frac{x-p_1}{q_1}$.
    % Falls $q_2 \neq 0$, setze $t := \displaystyle \frac{y-p_2}{q_2}$.}
  $\displaystyle t :=
  \begin{cases}
    \frac{x-p_1}{q_1} & \text{ falls } q_1 \neq 0 \\
    \frac{y-p_2}{q_2} & \text{ falls } q_2 \neq 0
  \end{cases}$
\item Mit $ax+by=c$ rechnet man nach, dass $(x,y) = P +tQ$.

  $\Rightarrow$ $(x,y) \in \ell$.\hfill$\Box$

% \item Seien $a,b,c\in\R$ mit $(a,b)\neq (0,0)$ und
%   $g=\{(x,y)\in\R^2\suchthat ax+by=c\}$.
% \item Es gelte $a\neq 0$ ($b\neq 0$ funktioniert analog).
% \item $P=(p_{1},p_{2}):=(c/a,0)$ und $Q=(q_{1},q_{2}):=(b,-a)$.
% \item $a\neq 0\Rightarrow (q_{1},q_{2})\neq (0,0)$.
% \item $\Rightarrow \ell :=\left\{P+t Q \suchthat t \in \R
%   \right\}$ ist eine Linie.
% \item Dann gilt $g=\ell$.
\end{itemize}

\end{document}
