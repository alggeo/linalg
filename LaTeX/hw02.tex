\documentclass[a4,11pt]{article}

\usepackage{exercises}
\usepackage{macros}

\usepackage{graphicx}
\usepackage{ngerman}

\vltitel{Lineare Algebra 2}
\dozent{\small{Christian Haase}}
\assistent{\small{Jan Marten Sevenster}}
\tutoren{\small{%
    Theresa Graeber \\[-1ex] Eva Schinzel}}

\semester{Sommersemester 2023%
  % \raisebox{-10mm}[0pt][0pt]{%
  %   \parbox{0pt}{\includegraphics[width=27mm]{../../2015-ana1-L/Vorlesungsmaterial/ana1QR}}}
}

\begin{document}
\vspace*{-17mm}
{
\kopf
}
% \vspace*{-5mm}
% \enlargethispage*{25mm}

\newcounter{chapter}
\ueblatt{2}{ Montag, 8.~Mai 2023 um 10h00 (ein PDF, jede Aufgabe auf neuer Seite beginnen)}

\begin{aufgabe}[4 Punkte]
Es sei $K$ ein K"orper. Ein Polynom $f\in K[t]$ heißt
\emph{irreduzibel}, wenn seine einzigen Faktorisierungen trivial sind,
d.h., dass f"ur $g,h \in K[t]$
\[
f = gh \ \text{ impliziert } \ \deg g = 0 \ \text{ oder } \ \deg h = 0.
\]
% wobei $K[t]^*$ die Menge der invertierbaren Elementen von $K[t]$ ist.

% \begin{enumerate}
% \item

Sei nun $f\in K[t]$ ein nicht-konstantes Polynom
vom Grad h"ochstens $3$. Zeigen Sie, dass $f$ irreduzibel ist, wenn es
keine Nullstellen hat.

% \item Geben Sie ein Polynom $f\in K[t]$ ohne Nullstellen an, welches
%   {\bfseries nicht} irreduzibel ist.
% \end{enumerate}
\end{aufgabe}

\begin{aufgabe}[4 Punkte]
Es sei $K$ ein K"orper. Wir definieren die $K$-lineare Abbildung $D
\colon K[t] \to K[t]$ auf der Basis der Monome via $D(t^k) := k\,t^{k-1}$
f"ur $k \in \Z_{\ge 0}$. 
  
\begin{enumerate}
\item Es sei $K$ ein K"orper der Charakteristik $0$ und $f \in K[t]$, zeigen Sie, dass f"ur $x\in K$ gilt
\[
\mu(f,x) = \max \left\{ n \in \mathbb{Z}_{\geq 0} \,\middle|\, f(x) = D(f)(x) =
D(D(f))(x) = \ldots = D^{n-1}(f)(x) = 0 \right\}.
\]
\item Geben Sie ein Beispiel an, woraus deutlich wird, dass diese Gleichheit nicht gilt f"ur $K = \Z/2\Z$.
\end{enumerate}

\end{aufgabe}

\begin{aufgabe}[4 Punkte]
Es sei $(F_n)_{n\geq 0}$ die bekannte Fibonacchi Folge gegeben durch
\[
F_0 = 0, \ F_1 = 1, \  F_{n} = F_{n-1} + F_{n-2}.
\]
Wir k"onnen diese Rekursion wie folgt formulieren:
\[
\begin{pmatrix}F_{n} \\ F_{n-1}\end{pmatrix} =
\begin{pmatrix} 1 & 1\\ 1 & 0\end{pmatrix}
\begin{pmatrix}F_{n-1} \\ F_{n-2}\end{pmatrix}.
\]
Wir setzen $A = \begin{pmatrix} 1 & 1\\ 1 & 0\end{pmatrix}$.
  
\begin{enumerate}
\item Diagonalisieren Sie die Matrix $A$.
\item Geben Sie eine geschlossene Formel f"ur $A^n$ an.
\item Geben Sie eine geschlossene Formel f"ur $F_n$ an.
\end{enumerate}

\end{aufgabe}

\end{document}

%%% Local Variables: 
%%% mode: latex
%%% End: 
