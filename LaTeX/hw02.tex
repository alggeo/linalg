\documentclass[a4,11pt]{article}

\usepackage{exercises}
\usepackage{macros}

\usepackage{graphicx}
\usepackage{ngerman}

\vltitel{Lineare Algebra 2}
\dozent{\small{Christian Haase}}
\assistent{\small{Jan Marten Sevenster}}
\tutoren{\small{%
    Theresa Graeber \\[-1ex] Eva Schinzel}}

\semester{Sommersemester 2023%
  % \raisebox{-10mm}[0pt][0pt]{%
  %   \parbox{0pt}{\includegraphics[width=27mm]{../../2015-ana1-L/Vorlesungsmaterial/ana1QR}}}
}

\begin{document}
\vspace*{-17mm}
{
\kopf
}
% \vspace*{-5mm}
% \enlargethispage*{25mm}

\newcounter{chapter}
\ueblatt{1}{ Montag, 8.~Mai 2023 um 10h00 (ein PDF, eine Seite pro Aufgabe)}

\begin{aufgabe}[4 Punkte]
Es sei $K$ ein K"orper und $f\in K[t]$ ein nicht konstantes Polynom vom Grad h"ochstens $3$. Ein Polynom heißt \emph{irreduzibel}, wenn seine einzigen Faktorisierungen trivial sind, d.h.
\[
f = gh \ \text{impliziert} \ g \in K[t]^* \ \text{oder} \ h\in K[t]^*,
\]
wobei $K[t]^*$ die Menge der invertierbaren Elementen von $K[t]$ ist.

Zeigen Sie, dass $f$ irreduzibel ist, wenn es keine Nullstellen hat.
\end{aufgabe}

\begin{aufgabe}[4 Punkte]
\begin{enumerate}
\item Es sei $K$ ein K"orper von Charakteristik $0$ und $f \in K[t]$, zeigen Sie, dass f"ur $x\in K$ gilt
\[
\mu(f,x) = \max \{ n \in \mathbb{Z}_{\geq 0} \mid  f^{(n)}(x) = 0 \}.
\]
\item Geben Sie ein Beispiel an, woraus deutlich wird, dass diese Gleichheit nicht gilt f"ur $K = \Z/2\Z$.
\end{enumerate}

\end{aufgabe}

\begin{aufgabe}[4 Punkte]
Es sei $(F_n)_{n\geq 0}$ die bekannte Fibonacchi Folge, gegeben durch
\[
F_0 = 0, \ F_1 = 1, \  F_{n} = F_{n-1} + F_{n-2}.
\]
Wir k"onnen diese Rekursion wie folgt formulieren:
\[
\begin{pmatrix}F_{n} \\ F_{n-1}\end{pmatrix} =
\begin{pmatrix} 1 & 1\\ 1 & 0\end{pmatrix}
\begin{pmatrix}F_{n-1} \\ F_{n-2}\end{pmatrix}.
\]
Wir setzen $A = \begin{pmatrix} 1 & 1\\ 1 & 0\end{pmatrix}$.
  
\begin{enumerate}
\item Diagonalisieren Sie die Matrix $A$.
\item Geben Sie eine geschlossene Formel f"ur $A^n$ an.
\item Geben Sie eine geschlossene Formel f"ur $F_n$ an.
\end{enumerate}

\end{aufgabe}

\end{document}

%%% Local Variables: 
%%% mode: latex
%%% End: 
