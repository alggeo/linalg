\documentclass[a4,11pt]{article}

\usepackage{exercises}
\usepackage{macros}

\usepackage{graphicx}
\usepackage{ngerman}

\vltitel{Lineare Algebra 2}
\dozent{\small{Christian Haase}}
\assistent{\small{Jan Marten Sevenster}}
\tutoren{\small{%
    Theresa Graeber \\[-1ex] Eva Schinzel}}

\semester{Sommersemester 2023%
  % \raisebox{-10mm}[0pt][0pt]{%
  %   \parbox{0pt}{\includegraphics[width=27mm]{../../2015-ana1-L/Vorlesungsmaterial/ana1QR}}}
}

\begin{document}
\vspace*{-17mm}
{
\kopf
}
% \vspace*{-5mm}
% \enlargethispage*{25mm}

\newcounter{chapter}
\ueblatt{1}{ Dienstag, 2.~Mai 2023 um 0h01 }

\begin{aufgabe}[4 Punkte]
  Beweise oder widerlege:
  \begin{enumerate}
  \item Jeder Endomorphismus $F \colon \R^2 \to \R^2$ hat einen
    Eigenvektor.
  \item Jeder Endomorphismus $F \colon \R^3 \to \R^3$ hat einen
    Eigenwert.
  \end{enumerate}
\end{aufgabe}

\begin{aufgabe}[4 Punkte]
Eigenwert / Diagonalisierbarkeit einer Matrix

\end{aufgabe}

\begin{aufgabe}[4 Punkte]
Sei $A$ n x n Matrix / K (look up Fischer Notation)
Angenommen,  alle Spaltensummen von $A$ sind gleich.
Zeigen Sie, dass  $A$ einen Eigenvektor hat.

\emph{Hinweis:} Was ist das charakteristische Polynom der
Transponierten?
\end{aufgabe}

\begin{aufgabe}[4 Punkte]
  Geben Sie zwei Matrizen an mit dem charakteristischen Polynom
  $(2-t)^2(3-t)$, so dass die eine diagonalisierbar ist und die andere
  nicht.
\end{aufgabe}

\end{document}

%%% Local Variables: 
%%% mode: latex
%%% TeX-master: t
%%% End: 
