\documentclass[a4,11pt]{article}

\usepackage{exercises}
\usepackage{macros}

\usepackage{graphicx}
\usepackage{ngerman}

\vltitel{Lineare Algebra 2}
\dozent{\small{Christian Haase}}
\assistent{\small{Jan Marten Sevenster}}
\tutoren{\small{%
    Theresa Graeber \\[-1ex] Eva Schinzel}}

\semester{Sommersemester 2023%
  % \raisebox{-10mm}[0pt][0pt]{%
  %   \parbox{0pt}{\includegraphics[width=27mm]{../../2015-ana1-L/Vorlesungsmaterial/ana1QR}}}
}

\begin{document}
\vspace*{-17mm}
{
\kopf
}
% \vspace*{-5mm}
% \enlargethispage*{25mm}

\newcounter{chapter}
\ueblatt{1}{ Dienstag, 2.~Mai 2023 um 0h01 }

\begin{aufgabe}[4 Punkte]
  Beweise oder widerlege:
  \begin{enumerate}
  \item Jeder Endomorphismus $F \colon \R^2 \to \R^2$ hat einen
    Eigenvektor.
  \item Jeder Endomorphismus $F \colon \R^3 \to \R^3$ hat einen
    Eigenwert.
  \end{enumerate}
\end{aufgabe}

\begin{aufgabe}[4 Punkte]
Es sei $H_2$ die Matrix
\[
H_2 = \begin{pmatrix}
1 & 1 & 1 & 1\\
1 & -1 & 1 & -1\\
1 & 1 & -1 & -1\\
1 & -1 & -1 & 1
\end{pmatrix}
\]
\begin{enumerate}
\item Bestimmen Sie die Eigenwerte von $H_2$.
\item Berechnen Sie für jeden Eigenwert eine Basis des zugeh"origen Eigenraums.
\item Ist $H_2$ diagonalisierbar?
\end{enumerate}

\emph{Hinweis:} Sie müssen sich nicht an diese Reihenfolge der
Aufgaben halten. Wenn Sie anhand der Struktur der Matrix einige
Eigenvektoren erkennen können, können Sie sich eine Menge Berechnungen
sparen.
\end{aufgabe}

\begin{aufgabe}[4 Punkte]
Sei $A \in M(n \times n ; K)$ eine $n \times n$ Matrix mit Eintr"agen in $K$.
Angenommen, alle Spaltensummen von $A$ sind gleich $s \in K$.

\begin{enumerate}
\item Rufen Sie sich die Definitionen der Begriffe \emph{Eigenvektor},
  \emph{Eigenwert}, \emph{charakteristisches Polynom} in Erinnerung.
\item Welche Beziehungen bestehen zwischen diesen Begriffen?
\item Geben Sie einen Eigenvektor von $\transpose{A}$ zum Eigenwert $s$ an.
\item Welche Beziehung besteht zwischen $\det A$ und $\det \transpose{A}$ ?
\item Welche Beziehung besteht zwischen $P_A$ und $P_{\transpose{A}}$ ?
\item Zeigen Sie, dass $A$ einen Eigenvektor hat.
\end{enumerate}

\end{aufgabe}

\begin{aufgabe}[4 Punkte]
  Geben Sie zwei Matrizen an mit dem charakteristischen Polynom
  $(2-t)^2(3-t)$, so dass die eine diagonalisierbar ist und die andere
  nicht.
\end{aufgabe}

\end{document}

%%% Local Variables: 
%%% mode: latex
%%% TeX-master: t
%%% End: 
