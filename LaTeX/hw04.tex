\documentclass[a4,11pt]{article}

\usepackage{exercises}
\usepackage{macros}

\usepackage{graphicx}
\usepackage{ngerman}

\vltitel{Lineare Algebra 2}
\dozent{\small{Christian Haase}}
\assistent{\small{Jan Marten Sevenster}}
\tutoren{\small{%
    Theresa Graeber \\[-1ex] Eva Schinzel}}

\semester{Sommersemester 2023%
  % \raisebox{-10mm}[0pt][0pt]{%
  %   \parbox{0pt}{\includegraphics[width=27mm]{../../2015-ana1-L/Vorlesungsmaterial/ana1QR}}}
}


% https://math.ug/la2-ss21/sec-cayley-hamilton.html


\begin{document}
\vspace*{-17mm}
{
\kopf
}
% \vspace*{-5mm}
% \enlargethispage*{25mm}

\newcounter{chapter}
\ueblatt{4}{ Montag, 22.~Mai 2023 um 10h00 (ein PDF, jede Aufgabe auf neuer Seite beginnen)}

\begin{aufgabe}[4 Punkte]
$I \subseteq K[t]$ Ideal, dann gibt es ein Polynom $p$ kleinsten
Grades in $I$.

Es gilt $I = (p)$.

Wenn $(p) = (q)$, dann gibt es ein $\lambda \in K \setminus \{0\}$ so
dass $q = \lambda p$.

\end{aufgabe}

\begin{aufgabe}[4 Punkte]
Wenn $F \circ G = G \circ F$, und $U$ $F$-invariant, dann sind $\ker G$
und $G(U)$ auch $F$-invariant.

Wenn zus"atzlich $F$ und $G$ diagonalisierbar, dann gibt es Basis aus
gemeinsamen Eigenvektoren
\end{aufgabe}


\begin{aufgabe}[4 Punkte]
Gebe Dimensionen der $\ker (F -\lambda)^i$ vor \ldots finde so ein $F$  
\end{aufgabe}

\end{document}

%%% Local Variables: 
%%% mode: latex
%%% End: 
