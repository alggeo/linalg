\documentclass[a4,11pt]{article}

\usepackage{exercises}
\usepackage{macros}

\usepackage{graphicx}
\usepackage{ngerman}

\vltitel{Lineare Algebra 2}
\dozent{\small{Christian Haase}}
\assistent{\small{Jan Marten Sevenster}}
\tutoren{\small{%
    Theresa Graeber \\[-1ex] Eva Schinzel}}

\semester{Sommersemester 2023%
  % \raisebox{-10mm}[0pt][0pt]{%
  %   \parbox{0pt}{\includegraphics[width=27mm]{../../2015-ana1-L/Vorlesungsmaterial/ana1QR}}}
}


% https://math.ug/la2-ss21/sec-cayley-hamilton.html


\begin{document}
\vspace*{-17mm}
{
\kopf
}
% \vspace*{-5mm}
% \enlargethispage*{25mm}

\newcounter{chapter}
\ueblatt{4}{ Montag, 22.~Mai 2023 um 10h00 (ein PDF, jede Aufgabe auf neuer Seite beginnen)}

\begin{aufgabe}[4 Punkte]
Es sei $K$ ein K\"orper und $0 \neq I \subseteq K[t]$ ein nicht triviales Ideal.

\begin{enumerate}
\item Zeigen Sie, dass es ein Polynom $0 \neq p \in I$ gibt, so dass
\[
\forall \ 0 \neq f \in I, \quad \degr (p) \leq \degr (f).
\]
\item Zeigen Sie weiter das, $I = (p)$.
\item Folgern Sie, dass es f\"ur ein zweites Polynom $q$, das diese Eigenschaft eirf\"ullt, einen Skalar $\lambda \in K \setminus \{0\}$ gibt, so dass $p = \lambda q$
\end{enumerate}

\end{aufgabe}

\begin{aufgabe}[4 Punkte]

Es seien $V$ ein Vektorraum und $F, G \in \text{End}(V)$ kommutierende Endomorphismen, d.h., $F \circ G = G \circ F$.
\begin{enumerate}
\item
Zeigen Sie, dass wenn $U \subset V$ ein $F$-invarianter Unterraum ist, $G(U)$ ebenfalls $F$-invariant ist, und dass auch $\ker G$  $F$-invariant ist.

\item Wir nehmen zus"atzlich an, dass  $F$ und $G$ diagonalisierbar sind und dass $V$ endlichdimensional ist, zeigen Sie dass es eine Basis $B$ von $V$ gibt, bestehend aus gemeinsame Eigenvektoren von $F$ und $G$. Also eine basis $B = \{ b_1, \dots , b_m\}$ f"ur die $b_i$ ein Eigenvektor von $F$ ist und ein Eigenvektor von $G$ f"ur alle $i = 1, \dots , m$, obwohl m"oglicherweise geh"orend zu verschiedenen Eigenwerten.
\end{enumerate}
\end{aufgabe}


\begin{aufgabe}[4 Punkte]
Es sei $V$ ein $7$ $K$-dimensionaler Vektorraum und $\lambda \in K$. Finden Sie ein Endomorphismus $F$, so dass
\begin{align*}
\dim \ker (F -\lambda I_V)^1 &= 3\\
\dim \ker (F -\lambda I_V)^2 &= 6 \\
\dim \ker (F -\lambda I_V)^3 &= 7
\end{align*}.
Geben Sie eine Darstellende Matrix f"ur $F$ an, und begr"unden Sie wie Sie die gefunden haben.
\end{aufgabe}

\end{document}

%%% Local Variables: 
%%% mode: latex
%%% End: 
