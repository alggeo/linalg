\documentclass[a4,11pt]{article}

\usepackage{./exercises}
\usepackage{./macros}
\usepackage{mathtools}
\usepackage{tikz-cd}

\usepackage{graphicx}
\usepackage{ngerman}

\vltitel{Lineare Algebra 2}
\dozent{\small{Christian Haase}}
\assistent{\small{Jan Marten Sevenster}}
\tutoren{\small{%
    Theresa Graeber \\[-1ex] Eva Schinzel}}

\semester{Sommersemester 2023%
  % \raisebox{-10mm}[0pt][0pt]{%
  %   \parbox{0pt}{\includegraphics[width=27mm]{../../2015-ana1-L/Vorlesungsmaterial/ana1QR}}}
}

\DeclareMathOperator{\End}{End}
\DeclareMathOperator{\Span}{span}
% \DeclareMathOperator{\ker}{ker}
\DeclareMathOperator{\im}{im}
\newcommand{\bonusitem}{\item\hspace*{-2.4mm}*\ }


\begin{document}
\vspace*{-17mm}
{
\kopf
}
% \vspace*{-5mm}
% \enlargethispage*{25mm}

\newcounter{chapter}
\ueblatt{11}{ Montag, 10.~Juli 2023 um 10h00}


\begin{aufgabe}[4 Punkte]
Die Deutsche Flagge kann auf die folgende Weise beschrieben werden
\[
F^D = \begin{pmatrix}
(0,0,0) & (0,0,0) & (0,0,0) & (0,0,0) & (0,0,0) \\
(255,0,0) & (255,0,0) & (255,0,0) & (255,0,0) & (255,0,0) \\
(255,204,0) & (255,204,0) & (255,204,0) & (255,204,0) & (255,204,0) \\
\end{pmatrix}
\]
Hier werden $3 \times 5 \times 3 = 45$ Zahlen gespeichert.

Berechnen Sie die (reduzierte) Singul"arwertzerlegungen von
\[
F^D_R = \begin{pmatrix}
0 & 0 & 0 & 0 & 0 \\
255 & 255 & 255 & 255 & 255 \\
255 & 255 & 255 &255 & 255 \\
\end{pmatrix}
\text{ und }
F^D_G = \begin{pmatrix}
0 & 0 & 0 & 0 & 0 \\
0 & 0 & 0 & 0 & 0 \\
204 & 204 & 204 &204 & 204 \\
\end{pmatrix}.
\]

Die Schweizer haben nur Wei"s und Rot in ihre Flagge. Die rote
Komponente hat also "uberall den Eintrag $255$ und die gr"une und
blaue sehen (ungef"ahr) so aus
\[
F^S_G = F^S_B = \begin{pmatrix}
0 & 0 & 0 & 0 & 0 \\
0 & 0 & 255 & 0 & 0 \\
0 & 255 & 255 & 255 & 0 \\
0 & 0 & 255 & 0 & 0 \\
0 & 0 & 0 & 0 & 0
\end{pmatrix}
\]
(ungef"ar, denn die L"angenverh"altnisse sind eigentlich etwas anders).
Berechnen Sie wieder die Singul"arwertzerlegung und beschreiben Sie,
wie die Schweizer Flagge aussehen w"urde, wenn L"ander nur Matrizen
vom Rang $1$ benutzen d"urften, um ihre Flagge zu kodieren, und die
Schweizer die (in Frobeniusnorm) n"achstgelegenen Rang-eins-Matrizen
n"ahmen, um ihre wirkliche Flagge zu kodieren.

Sie k"onnen dazu entweder ein Bild oder Matrizen angeben.
\end{aufgabe}

\begin{aufgabe}[4 Punkte]
Zeigen Sie, dass die Singul"arwerte einer Matrix gleich bleiben, wenn
die Matrix von links oder von rechts mit einer unit"aren Matrix
multipliziert wird. Dass hei"st, wenn $U$ und $V$ unit"ar von
geeigneter Gro"se sind, dann haben
\[
M, \ UM, \text{ und } MV
\]
alle die gleiche Singul"arwerte.
\end{aufgabe}


\begin{aufgabe}[4 Punkte]
Finden Sie eine reelle oder komplexe Matrix die diagonalisierbar ist
aber nicht normal.
\end{aufgabe}

\begin{aufgabe}[4 Quizbonuspunkte]
  Fertigen Sie eine Mindmap zum Thema
  Skalarprodukte
  an:
  \begin{enumerate}
  \item Sammeln Sie alle Definitionen, Resultate (S"atze, Lemmata,
    Korollare, Beobachtungen, \ldots) und Beispiele.
  \item Ordnen Sie die Definitionen, Resultate und Beispiele
    thematisch auf einer Mindmap an.
  \item Verbinden Sie Items, die zueinander in direkter Beziehung
    stehen (Definition wird im Beweis vom Satz benutzt, Lemma ist
    Spezialfall von Proposition, \ldots).
  \end{enumerate}
%  Alternativ k"onnen Sie auch (1) auf Karteikarten notieren und diese 
\end{aufgabe}


% \footnotetext{*: Ggf.\ im Tutorium besprechen.}


\end{document}

%%% Local Variables: 
%%% mode: latex
%%% End: 
