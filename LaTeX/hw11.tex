\documentclass[a4,11pt]{article}

\usepackage{./exercises}
\usepackage{./macros}
\usepackage{mathtools}
\usepackage{tikz-cd}

\usepackage{graphicx}
\usepackage{ngerman}

\vltitel{Lineare Algebra 2}
\dozent{\small{Christian Haase}}
\assistent{\small{Jan Marten Sevenster}}
\tutoren{\small{%
    Theresa Graeber \\[-1ex] Eva Schinzel}}

\semester{Sommersemester 2023%
  % \raisebox{-10mm}[0pt][0pt]{%
  %   \parbox{0pt}{\includegraphics[width=27mm]{../../2015-ana1-L/Vorlesungsmaterial/ana1QR}}}
}

\DeclareMathOperator{\End}{End}
\DeclareMathOperator{\Span}{span}
% \DeclareMathOperator{\ker}{ker}
\DeclareMathOperator{\im}{im}
\newcommand{\bonusitem}{\item\hspace*{-2.4mm}*\ }


\begin{document}
\vspace*{-17mm}
{
\kopf
}
% \vspace*{-5mm}
% \enlargethispage*{25mm}

\newcounter{chapter}
\ueblatt{11}{ Montag, 10.~Juli 2023 um 10h00}


\begin{aufgabe}[4 Punkte]
Die Deutsche Flagge kann auf die folgende Weise beschrieben werden
\[
F^D = \begin{pmatrix}
(0,0,0) & (0,0,0) & (0,0,0) & (0,0,0) & (0,0,0) \\
(255,0,0) & (255,0,0) & (255,0,0) & (255,0,0) & (255,0,0) \\
(255,204,0) & (255,204,0) & (255,204,0) & (255,204,0) & (255,204,0) \\
\end{pmatrix}
\]
Hier werden $3 \times 5 \times 3 = 45$ zahlen gespeichert.

Berechnen Sie die  (reduzierte) Singul"arwertzerlegungen von
\[
F^D_R = \begin{pmatrix}
0 & 0 & 0 & 0 & 0 \\
255 & 255 & 255 & 255 & 255 \\
255 & 255 & 255 &255 & 255 \\
\end{pmatrix}
\text{ und }
F^D_G = \begin{pmatrix}
0 & 0 & 0 & 0 & 0 \\
0 & 0 & 0 & 0 & 0 \\
204 & 204 & 204 &204 & 204 \\
\end{pmatrix}.
\]

Die Schweizer habe nur wei"s und rot in ihre Flagge. Die rote Komponente hat also "uberal $255$ und die gr"une un blaue sehen (ungef"ahr) so aus
\[
F^S_G = F^S_B = \begin{pmatrix}
0 & 0 & 0 & 0 & 0 \\
0 & 0 & 255 & 0 & 0 \\
0 & 255 & 255 & 255 & 0 \\
0 & 0 & 255 & 0 & 0 \\
0 & 0 & 0 & 0 & 0
\end{pmatrix}.
\]
Ungef"ar denn die verh"altnisse eigentlich etwas anders sind. Aber wir nehme an, dass diese verh"altnisse stimmen.
Berechnen Sie wieder die Singul"arwertzerlegung und beschreiben wie die Schweizer Flagge aussehen w"urde, wenn L"ander nur Matrizen des Rangs $1$ benutzen d"urften um ihre Flagge zu kodieren und die Schweizer die n"achstgelegene (im Frobeniusnorm) rang eins Matrizen nehmen um ihre wirkliche Flagge zu kodieren. Sie k"onnen dazu entweder ein Bild oder Matrizen angeben.
\end{aufgabe}

\begin{aufgabe}[4 Punkte]
Zeigen Sie, dass die Singul"arwerte einer Matrix gleich bleiben, wenn die Matrix von links oder rechts mit einer Unit"aren Matrix multipliziert wird. Dass hei"st, f"ur $U, V$ unt"ar der geeignete gro"se, haben
\[
M, \ UM, \text{ und } MV
\]
alle die gleiche Eigenwerte.
\end{aufgabe}


 \begin{aufgabe}[4 Punkte]
Suchen Sie eine reelle oder komplexe Matrix die diagonalisierbar ist aber nicht normal.
 \end{aufgabe}


% \footnotetext{*: Ggf.\ im Tutorium besprechen.}


\end{document}

%%% Local Variables: 
%%% mode: latex
%%% End: 
