\documentclass[a4paper]{article}

\usepackage{mdframed}
\usepackage{tikz}
\usepackage{pdfpages}
% \usetikzlibrary{shadows}
% \usetikzlibrary{calc}
\usetikzlibrary{patterns}
	\usetikzlibrary{intersections,calc,decorations.pathreplacing}
\newcommand{\ngon}[5]{ % #1 = number of vertices
                       % #2 = center point of the n-gon
                       % #3 = diameter of the circle defining the n-gon
                       % #4 = label name of the object to use relative coordinates
                       % #5 = rotation

  \foreach \t in {1,...,#1} {
    \coordinate (#4\t) at ($#2+(#5-\t*360/#1:#3)$);
    \fill (#4\t) circle (5pt);
  }
}

\usepackage{me-exercises}
\usepackage{graphicx}
\usepackage{ngerman}
\usepackage{entdecken-macros}
% \graphicspath{{../../texmf/Bilder/}}
% \definecolor{graphcolor}{rgb}{.7,0,0}
% \definecolor{cryptcolor}{rgb}{0,0,.7}
% \definecolor{codescolor}{rgb}{0,.4,0}
% \definecolor{darkgreen}{rgb}{0,0,0}
% \newcommand{\gr}{\color{graphcolor}}
% \newcommand{\cy}{\color{cryptcolor}}
% \newcommand{\er}{\color{codescolor}}


\tikzset{
  schraffiert/.style={pattern=north west lines,pattern color=#1},
  schraffiert/.default=black}

\newcommand{\sage}{%
  \raisebox{-.6mm}{\includegraphics[width=9mm]{../qsl/dayOne/sagenb}}}


\vltitel{Lineare Algebra 2
  \makebox[1mm]{\raisebox{-19mm}[0pt][0pt]{%
      \hspace*{4mm}
 %     \includegraphics[width=25mm]{../qsl/abacus}
}}}
\dozent{Christian Haase}
\assistent{Jan Sevenster}
\tutoren{Theresa Graeber \\[-1ex] Eva Schinzel}
\semester{Sommersemester 2023}

% \newcommand{\N}{\mathbb{N}}
% \newcommand{\Z}{\mathbb{Z}}
% \newcommand{\R}{\mathbb{R}}
% \newcommand{\C}{\mathbb{C}}
% \newcommand{\Q}{\mathbb{Q}}

% \DeclareMathOperator{\Spur}{Spur}

% \theorembodyfont{\rmfamily}
% \newtheorem{aufgabe}{Aufgabe}
\pagestyle{empty}
%\geometry{body={16cm,22cm},left=25mm}

\setlength{\parindent}{0pt}

\semester{Wintersemester 2021/22}

% \renewcommand{\pagebreak}{\vspace{8mm} \par}

\begin{document}
% \includepdf{Klausurdeckblatt.pdf}

{
\kopf
}
 \begin{center}
   {\Large {\bfseries\scshape Abschlussklausur}}
\end{center}

\enlargethispage*{40mm}

Bearbeiten Sie bitte die Aufgaben 1-4 und w\"ahlen Sie {\bfseries
  \sffamily EINE} der Aufgaben 5 oder 6. Markieren Sie bitte unten,
welche benotet werden soll. 
\\
Bitte nehmen Sie sich Zeit, die Aufgabenstellung durchzulesen und zu
verstehen. Begr\"unden Sie Ihre Antworten!
\\
Schreiben Sie bitte Ihre L\"osungen auf diese Bl\"atter. 
Sollten die Vorder- und R\"uckseiten nicht ausreichen, stellen wir
Schmierpapier zur Verf\"ugung.
Bitte schreiben Sie Ihren Namen und Ihre Matrikelnummer auf diese
Extraseiten. 
\\
Tragen Sie bitte jetzt Ihren Namen und Ihre Matrikelnummer ein.
Bitte bl\"attern Sie erst um, wenn zum Angriff geblasen wird.
\\[\baselineskip] %\vfill
Denken Sie nach, bleiben Sie ruhig. Viel Spa"s und Erfolg.
\\[\baselineskip] %\vfill
Name: \underline{\hspace{50mm}} \hfill
\mbox{Matrikel: \underline{\hspace{40mm}}}
\\[.5\baselineskip]
\mbox{} \hfill \mbox{Studiengang: \underline{\hspace{40mm}}}
\\[\baselineskip] %\vfill
\begin{center}
  \begin{tabular}{l c | rcr}
    \multicolumn{2}{r|}{Aufgabe} & Punkte \\
    \hline
    $\boxtimes$ & 1 & & / & 8 \\
    $\boxtimes$ & 2 & & / & 8 \\
    $\boxtimes$ & 3 & & / & 8 \\
    $\boxtimes$ & 4 & & / & 8 \\
%    $\boxtimes$ & 5 & & / & 8 \\
%    $\Box$ & 5 & & / & 10 \\
    $\Box$ & 5 & & / & 8 \\
    $\Box$ & 6 & & / & 8 \\
%    $\Box$ & 8 & & / & 7 \\
    \hline
    & $\sum$ & & / & 40
  \end{tabular}
\end{center}
\vfill

\vspace{5mm}

\begin{center}
  % \includegraphics[width=.7\textwidth]{massa-marianneth-fotocumunity}
  % \\
  % \Large Denken ist wie {\tt googeln}, nur krasser.
%%\includegraphics[width=.7\textwidth]{bunnyTee}
%%\\
%\mbox{} \hfill {\tiny The Book of Bunny Suicides \copyright Andy Riley}
\end{center}
%% \vfill \vfill
%\mbox{}

\newpage

AUFGABENSAMMLUNG

\begin{klaufg}{8}{}
beweise oder widerlege: zu jeder Matrix gibt es ein Skalarprodukt, so dass sie unit"ar ist
\end{klaufg}

\begin{klaufg}{8}{}
sei $A$ eine Matrix, so dass $A^{12} = E_n$. beweise oder widerlege: es gibt ein Skalarprodukt, so dass $A$ unit"ar ist.
\end{klaufg}

\begin{klaufg}{8}{}
\begin{enumerate}
\item
Bestimmen die Anzahl der Matrizen $M \in \operatorname{Mat}(2\times 2; \mathbb{F}_2)$ mit dem charakteristischen Polynom 
\[
p_M = x^2 + x + 1
\]
\item
Bestimmen die Anzahl der Matrizen $M \in \operatorname{Mat}(2\times 2; \mathbb{F}_2)$ mit dem Minimalpolynom 
\[
m_M = x^2 + x + 1
\]
\item
Ist die Matrix
\[
\begin{pmatrix}
1 & 0 & 0\\
0 & 0 & 1\\
0 & 1 & 1
\end{pmatrix} \in \operatorname{Mat}(3\times 3; \mathbb{F}_2)
\]
diagonalisierbar?
\end{enumerate}
\end{klaufg}

\begin{klaufg}{8}{}
Geben Sie ein Beispiel einer reellen nilpotenten Matrix, die nicht eine obere oder untere Dreieicksmatrix ist.
\end{klaufg}

\begin{klaufg}{8}{}
Bestimmen Sie f"ur $n > 0$ die Jordansche Normalform von
\[
\begin{pmatrix}
1 & 1 & \cdots & 1\\
1 & 1 &  & 1\\
\vdots & & \ddots & 1 \\
1 & 1&  \cdots & 1
\end{pmatrix} \in \operatorname{Mat}(n \times n; \mathbb{R})
\]
\end{klaufg}


\pagebreak

\begin{klaufg}{4+4}{Reproduktion}
  Erkl"aren Sie {\bfseries\sffamily kurz},

\begin{enumerate} %[(i)]
\item[(\it i)] ...
   \vfill
\item[(\it ii)] ...
 \vfill
\end{enumerate}

\end{klaufg}

\pagebreak
\begin{klaufg}{4+4}{examples}

  \makebox[0pt][r]{($i$) }%

  Geben Sie Beispiele von ...

  (Begr"unden Sie {\bfseries\sffamily kurz}, dass Ihr Beispiel die
  geforderten Eigenschaften hat.)
  \pagebreak
  
  \makebox[0pt][r]{($ii$) }%
    Geben Sie Beispiele von ...
  
  (Begr"unden Sie {\bfseries\sffamily kurz}, dass Ihr Beispiel die
  geforderten Eigenschaften hat.)

\end{klaufg}

\pagebreak
\begin{klaufg}{8}{einfacher Beweis}

\end{klaufg}

\pagebreak
\begin{klaufg}{8}{einfache Modellierung/Beweis}

  
\end{klaufg}
 
\pagebreak
{\bfseries Nur eine der Aufgaben 5 \& 6 wird gewertet.}
\bigskip
\bigskip

\begin{klaufg}{8}{hard1}

\end{klaufg}

\pagebreak
{\bfseries Nur eine der Aufgaben 5 \& 6 wird gewertet.}
\bigskip
\bigskip

\begin{klaufg}{8}{hard2}

\end{klaufg}

\pagebreak
NACHKLAUSUR

\begin{klaufg}{4+4}{Reproduktion}
  Erkl"aren Sie {\bfseries\sffamily kurz},

\begin{enumerate} %[(i)]
\item[(\it i)] ...
   \vfill
\item[(\it ii)] ...
 \vfill
\end{enumerate}

\end{klaufg}

\pagebreak
\begin{klaufg}{4+4}{examples}

  \makebox[0pt][r]{($i$) }%

  Geben Sie Beispiele von ...

  (Begr"unden Sie {\bfseries\sffamily kurz}, dass Ihr Beispiel die
  geforderten Eigenschaften hat.)
  \pagebreak
  
  \makebox[0pt][r]{($ii$) }%
    Geben Sie Beispiele von ...
  
  (Begr"unden Sie {\bfseries\sffamily kurz}, dass Ihr Beispiel die
  geforderten Eigenschaften hat.)

\end{klaufg}

\pagebreak
\begin{klaufg}{8}{einfacher Beweis}

\end{klaufg}

\pagebreak
\begin{klaufg}{8}{einfache Modellierung/Beweis}

  
\end{klaufg}
 
\pagebreak
{\bfseries Nur eine der Aufgaben 5 \& 6 wird gewertet.}
\bigskip
\bigskip

\begin{klaufg}{8}{hard1}

\end{klaufg}

\pagebreak
{\bfseries Nur eine der Aufgaben 5 \& 6 wird gewertet.}
\bigskip
\bigskip

\begin{klaufg}{8}{hard2}

\end{klaufg}



\end{document}




