\documentclass[a4paper]{article}


\usepackage{amsmath,amssymb,amsthm}
\usepackage{mathtools}


\usepackage{ngerman}

\title{Inhaltsverzeichnis Lineare Algebra II SoSe2023}
\begin{document}
\maketitle

\textbf{Diese Liste ist nicht vollst"andig. Sie ist nur ein Hilfsmittel und aus diesem Dokument können keine Rechte abgeleitet werden}

\begin{enumerate}
\item
Charakterischische Polynomen, Minimalpolynomen, Eigenwerte, Eigenvektoren, Diagonalisierbarkeit, Trigonalisierbarkeit, Jordansche Normalformen.

Ben"otigde F"ahigkeiten:
\begin{enumerate}
\item
Charakterischische Polynomen, Minimalpolynomen, Eigenwerte und Eigenr"aume von Matrizen und Endomorphismen ausrechnen.
\item
Matrizen trigonalisieren, oder Diagonalisieren, wenn m"oglich. Jordansche Normalformen ausrechnen.
\item
Geeignete Basen bestimmen, in Bezug auf die die darstellende Matrix Diagonal-, Trigonal- oder Jordanform annimmt.
\item
Die Korrespondenz zwischen $\text{End}(V)$ und $\text{Mat}(\dim V \times \dim V)$ verstehen und Eigenschaften der Endomorphismen "ubersetzen in Eigenschaften der Matrizen.
\item
Eigenschaften von Matrizen und Endomorphismen und Matrizen in Verbingung bringen mit Determinanten, charakteristische Polynomen, Dimensionen der Eigenr"aume, usw.
\end{enumerate}
Literaturhinweise:

\cite{FS} Kaptel $5$, oder vielleicht $4$, abh"angig von Ihre Version. Der Abschnitt $5.6$ aus \cite{FS} wurde vernachl"assigt zu Gunsten des Abschnitts 16.3 in \cite{G}. Es handelt sich dort um den Beweis des Satzes des Cayley-Hamilton.
\item
Bilinearformen, Skalarprodukten, Orthogonalit"at, Normen.

Ben"otigde F"ahigkeiten:
\begin{enumerate}
\item
Darstellende Matrizen bestimmen f"ur Bilinearformen.
\item
Kriterien f"ur Skalarprodukten nennen und ableiten, ob gegebene Bilinearformen diese erf"ullen.
\item
Orthogonalbasen bestimmen aus gegebenen Basen und f"ur endlichdimensionalen Vektorr"aumen ohne spezifizierten Basis.
\item
Cauchy-Bunyakovsky-Schwarz Ungleichung und ihre Folgerungen anwenden.
\item
Adjungierte Abbildungen bestimmen und ihre Eigenschaften in Rechnungen benutzen.
\item
Eigenschaften von Selbstadjungierte, symmetrische und schiefsymmetrische Abbildungen kennen und anwenden.
\item
Rechnen mit Normen, insbesondere Frobeniusnorm, Standardnorm und Spektralnorm und senkrechte Projektionen.
\end{enumerate}

Literaturhinweise:

\cite{FS} Kapitel $6$, insbesondere $6.4$ bis zum $6.7$, $6.7.7$ ($19$. Auflage). F"ur die Definitionen der Frobenius- und Spektralnorm k"onnen Sie die Quizfragen nachschlagen. Sie sind nicht explizit so genannt, aber ihre Definitionen w"urden gegebenenfalls auf der Klausur in Erinnerung gerufen.

\item
Singul"arwertzerlegungen

Ben"otigde F"ahigkeiten:
\begin{enumerate}
\item
Singul"arwertzerlegung einer Matrix ausrechnen.
\item
Eigenschaften ableiten von Matrizen, anhand der Singul"arwerten.
\end{enumerate}

Literaturhinweise:

\cite{G}, Abschnitt $19.8.1$.

\item
Google Pagerank

Ben"otigde F"ahigkeiten:
\begin{enumerate}
\item
Die Beziehung zwischen Internets (gerichteten Graphen) und Matrizen verstehen.
\item
Bedingungen kennen unter denen Matrizen Eigenvektoren zum Eigenwert $1$ haben, und der Eigenraum zum Eigenwert $1$ einsdimensional ist.
\item
Die Problemen mit dem naiven Pagerankverfahren und m"ogliche L"osungen verstehen (\cite{BL} $2.2$).
\end{enumerate}

Literaturhinweise:

\cite{BL}.

\item
Darstellungstheorie

\end{enumerate}

\begin{thebibliography}{10}
\bibitem{FS}
G.Fischer, Springborn, \emph{Lineare Algebra}, Springer
\bibitem{G}
U. G"ortz, \emph{Lineare Algebra 2}, online Vorlesung
\bibitem{BL}
K. Bryan, T. Leise, \emph{The \$$25,000,000,000$ Eigenvector, The Linear Algebra Behind Google}
\end{thebibliography}

\end{document}