\documentclass[a4,11pt]{article}

\usepackage{exercises}
\usepackage{macros}
\usepackage{mathtools}
\usepackage{tikz-cd}

\usepackage{graphicx}
\usepackage{ngerman}

\vltitel{Lineare Algebra 2}
\dozent{\small{Christian Haase}}
\assistent{\small{Jan Marten Sevenster}}
\tutoren{\small{%
    Theresa Graeber \\[-1ex] Eva Schinzel}}

\semester{Sommersemester 2023%
  % \raisebox{-10mm}[0pt][0pt]{%
  %   \parbox{0pt}{\includegraphics[width=27mm]{../../2015-ana1-L/Vorlesungsmaterial/ana1QR}}}
}

\DeclareMathOperator{\End}{End}

\begin{document}
\vspace*{-17mm}
{
\kopf
}
% \vspace*{-5mm}
% \enlargethispage*{25mm}

\newcounter{chapter}
\ueblatt{8}{ Montag, 19.~Juni 2023 um 10h00}


\begin{aufgabe}[4 Punkte]
Wir betrachten die Vektoren $1,x,x^2 \in C^0([0,2],\mathbb{R})$, wobei $C^0([0,2],\mathbb{R})$ den Vektorraum der stetigen reellwertigen Funktionen auf $[0,2]$ bezeichnet. Dieser Vektorraum ist mit dem folgenden Skalarprodukt ausgestattet:
\[
\left<f,g\right> = \int_0^2 f(x)g(x)dx.
\]
Bestimmen Sie mit dem Schmidtsen Verfahren eine Orthonormalbasis von $\text{span}(1,x,x^2)$.
\end{aufgabe}

\begin{aufgabe}[4 Punkte]
\begin{enumerate}
\item Es sei $V$ ein endlichdimensionaleVektorraum "uber einen K"orper $K$ mit einem Skalarprodukt $\left<-,-\right>$. Zeigen Sie, dass $U^{\perp \perp} = U$
\item Geben Sie ein Beispiel an f"ur einen Vektorraum mit Skalarprodukt, f"ur den diese Gleichheit nicht gilt.
\item Welche der Inklusionen $U^{\perp \perp} \subseteq U$  und $U \subseteq U^{\perp \perp}$ gilt auch in undlichdimensionale Vektorr"aume?
\end{enumerate}
\end{aufgabe}

\begin{aufgabe}[4 Punkte]
Es sei $M \in \text{Mat}(n \times n; \mathbb{C})$ eine Hermitesche Matrix, zeigen Sie, dass die folgenden Aussagen "aquivalent sind:
\begin{enumerate}
\item $M$ ist positiv semidefinit
\item Es gibt eine Matrix $W \in \text{Mat}(n\times n;\mathbb{C})$ f"ur welche $M = \overline{\transpose \ W} W$.
\end{enumerate}
\end{aufgabe}

\end{document}

%%% Local Variables: 
%%% mode: latex
%%% End: 
