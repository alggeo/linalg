\documentclass[a4,11pt]{article}

\usepackage{exercises}
\usepackage{macros}
\usepackage{mathtools}
\usepackage{tikz-cd}

\usepackage{graphicx}
\usepackage{ngerman}

\vltitel{Lineare Algebra 2}
\dozent{\small{Christian Haase}}
\assistent{\small{Jan Marten Sevenster}}
\tutoren{\small{%
    Theresa Graeber \\[-1ex] Eva Schinzel}}

\semester{Sommersemester 2023%
  % \raisebox{-10mm}[0pt][0pt]{%
  %   \parbox{0pt}{\includegraphics[width=27mm]{../../2015-ana1-L/Vorlesungsmaterial/ana1QR}}}
}

\DeclareMathOperator{\End}{End}

\begin{document}
\vspace*{-17mm}
{
\kopf
}
% \vspace*{-5mm}
% \enlargethispage*{25mm}

\newcounter{chapter}
\ueblatt{8}{ Montag, 19.~Juni 2023 um 10h00}


\begin{aufgabe}[4 Punkte]
orthonormalisiere $1,x,x^2$ auf $[0,2]$
\end{aufgabe}

\begin{aufgabe}[4 Punkte]
\begin{enumerate}
\item $\dim V < \infty$ dann $U^{\perp \perp} = U$
\item Beispiel, dass $U^{\perp \perp} \neq U$
\item welche Inklusion gilt immer?
\end{enumerate}
\end{aufgabe}

\begin{aufgabe}[4 Punkte]
  irgendwas zu positiv (semi)definit

  Wurzel?
\end{aufgabe}

\end{document}

%%% Local Variables: 
%%% mode: latex
%%% End: 
