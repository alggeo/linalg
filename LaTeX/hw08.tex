\documentclass[a4,11pt]{article}

\usepackage{exercises}
\usepackage{macros}
\usepackage{mathtools}
\usepackage{tikz-cd}

\usepackage{graphicx}
\usepackage{ngerman}

\vltitel{Lineare Algebra 2}
\dozent{\small{Christian Haase}}
\assistent{\small{Jan Marten Sevenster}}
\tutoren{\small{%
    Theresa Graeber \\[-1ex] Eva Schinzel}}

\semester{Sommersemester 2023%
  % \raisebox{-10mm}[0pt][0pt]{%
  %   \parbox{0pt}{\includegraphics[width=27mm]{../../2015-ana1-L/Vorlesungsmaterial/ana1QR}}}
}

\DeclareMathOperator{\End}{End}
\DeclareMathOperator{\Span}{span}
\newcommand{\bonusitem}{\item\hspace*{-2.4mm}*\ }


\begin{document}
\vspace*{-17mm}
{
\kopf
}
% \vspace*{-5mm}
% \enlargethispage*{25mm}

\newcounter{chapter}
\ueblatt{8}{ Montag, 19.~Juni 2023 um 10h00}


\begin{aufgabe}[4 Punkte]
Wir betrachten die Vektoren $1,x,x^2 \in C^0([0,2],\mathbb{R})$, wobei $C^0([0,2],\mathbb{R})$ den Vektorraum der stetigen reellwertigen Funktionen auf $[0,2]$ bezeichnet. Dieser Vektorraum ist mit dem folgenden Skalarprodukt ausgestattet:
\[
\left<f,g\right> := \int_0^2 f(x)g(x)dx.
\]
(Sie brauchen nicht nachzuweisen, dass es sich um ein Skalarprodukt
handelt.)

Bestimmen Sie mit dem Gram-Schmidtschen Verfahren eine Orthonormalbasis von $\Span(1,x,x^2)$.
\end{aufgabe}

\begin{aufgabe}[4 Punkte]
\begin{enumerate}
\item Es sei $(V,\left<-,-\right>)$ ein endlichdimensionaler
  euklidischer oder unit"arer Vektorraum. Zeigen Sie, dass $U^{\perp
    \perp} = U$.
\item Welche der Inklusionen $U^{\perp \perp} \subseteq U$ oder $U
  \subseteq U^{\perp \perp}$ gilt in allen euklidischen oder unit"aren
  Vektorr"aumen?
\item Geben Sie ein Beispiel an f"ur einen Unterraum eines
  euklidischen Vektorraums, f"ur den diese Gleichheit nicht gilt.
\end{enumerate}
\end{aufgabe}

\begin{aufgabe}[4 Punkte]
Sei $n \in \Z_{>0}$ und $A \in M(n \times n; \R)$ symmetrisch mit der
Eigenschaft, dass f"ur alle $i=1, \ldots, n$
\begin{equation} \tag{$*$}
a_{ii} > \sum_{j \neq i} |a_{ij}| \,.
\end{equation}
\begin{enumerate}
\item Zeigen Sie, dass im Fall $n=2$ f"ur $A = \left(
    \begin{smallmatrix}
      a&b\\b&c
    \end{smallmatrix} \right)$ und $v = \binom{x}{y}$ die Absch"atzung
  $$
  ^t\!v A v \ge |b| (|x|-|y|)^2
  $$
  gelten muss.
\item Folgern Sie, dass (immernoch $n=2$) die Matrix $A$ positiv
  definit ist.
\item Beweisen Sie f"ur allgemeines $n$, dass die Eigenschaft $(*)$
  impliziert, dass $A$ positiv definit ist.
\bonusitem Geben Sie ein Beispiel (mit Begr"undung) einer positiv
definiten Matrix, die Eigenschaft $(*)$ verletzt.
\end{enumerate}
\footnotetext{*: Ggf.\ im Tutorium besprechen.}
\end{aufgabe}

% \begin{aufgabe}[4 Punkte]
% Es sei $M \in \text{Mat}(n \times n; \mathbb{C})$ eine Hermitesche Matrix, zeigen Sie, dass die folgenden Aussagen "aquivalent sind:
% \begin{enumerate}
% \item $M$ ist positiv semidefinit
% \item Es gibt eine Matrix $W \in \text{Mat}(n\times n;\mathbb{C})$ f"ur welche $M = \overline{\transpose \ W} W$.
% \end{enumerate}
% \end{aufgabe}

\end{document}

%%% Local Variables: 
%%% mode: latex
%%% End: 
