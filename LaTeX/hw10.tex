\documentclass[a4,11pt]{article}

\usepackage{./exercises}
\usepackage{./macros}
\usepackage{mathtools}
\usepackage{tikz-cd}

\usepackage{graphicx}
\usepackage{ngerman}

\vltitel{Lineare Algebra 2}
\dozent{\small{Christian Haase}}
\assistent{\small{Jan Marten Sevenster}}
\tutoren{\small{%
    Theresa Graeber \\[-1ex] Eva Schinzel}}

\semester{Sommersemester 2023%
  % \raisebox{-10mm}[0pt][0pt]{%
  %   \parbox{0pt}{\includegraphics[width=27mm]{../../2015-ana1-L/Vorlesungsmaterial/ana1QR}}}
}

\DeclareMathOperator{\End}{End}
\DeclareMathOperator{\Span}{span}
% \DeclareMathOperator{\ker}{ker}
\DeclareMathOperator{\im}{im}
\newcommand{\bonusitem}{\item\hspace*{-2.4mm}*\ }


\begin{document}
\vspace*{-17mm}
{
\kopf
}
% \vspace*{-5mm}
% \enlargethispage*{25mm}

\newcounter{chapter}
\ueblatt{9}{ Montag, 26.~Juni 2023 um 10h00}


\begin{aufgabe}[4 Punkte]
Es sei $V$ ein endlichdimensionaler Skalarproduktraum mit dem Skalarprodukt $\left<\cdot , \cdot \right>$, und nehme $U \subseteq V$ einen Unterraum.
  
  \begin{enumerate}
  \item Geben Sie zu einem Vektor $b \in V$, einen Vektor $u \in U$ an, so dass  $b-u \perp U$. Zeigen Sie, dass es genau einen solchen Vektor in $U$ gibt.
  \item Zeigen Sie, dass das Endomorphismus $b \mapsto u$, das einem Vektor $b \in V$ diesen Vektor $u$ zuweist, linear ist.
  \item Der Vektor $u$ wird dadurch charakterisiert, dass 
  \[
  \| b - u \|^2 = \min_{w\in U} \| b - w \|^2.
  \]
  Natürlich ist $\| \cdot \|$ hier der durch $\left<\cdot , \cdot \right>$ induzierte Norm. Beweisen Sie diese Gleichung.
\end{enumerate}
\end{aufgabe}

\begin{aufgabe}[4 Punkte]
  Wie nehmen $F \colon V \to W$ ein injektives Homomorphismus zwischen endlichdimensionalen Skalarproduktr"uamen.
  \begin{enumerate}
  \item Zeigen Sie, dass $F^*F$ invertierbar ist.
  \item Gegeben ist ein Vektor $b \in W$, geben Sie eine Formel f"ur $v \in V$ an, so dass $\| b - F(v) \|^2$ minimiert wird "uber $V$, d.h.
  \[
  \| b - F(v) \|^2 = \min_{w\in U} \| b - F(w) \|^2.
  \]
  \item Wir haben eine Reihe Daten $(1,y_1), \ldots, (n,y_n) \in \mathbb{R}^2$ gemessen. Suchen Sie zu einem Polynom $p = b_0 + b_1 x + b_2 x^2\in \mathbb{R} [x]_{\leq 2}$ eine lineare Abbildung $P : \mathbb{R}^3 \rightarrow \mathbb{R}^n$ die
  \[
  P(\transpose (b_0 \ b_1 \ b_2)) = \transpose (p(1) \ p(2) \dots p(n))
  \]
  erf"ullt.
   Bestimmen Sie mit Hilfe dieser Abbildung und den vorigen Teilfragen ein quadratisches Polynom $f$, so dass
  \[
  \sum_{i = 1}^n (f(i)-y_i)^2 = \min_{g \in \mathbb{R} [x]_{\leq 2}} \sum_{i = 1}^n (g(i)-y_i)^2 .
  \]
  \end{enumerate}
\end{aufgabe}


 \begin{aufgabe}[4 Punkte]
 Es sei $M \in \text{Mat}(n \times n; \mathbb{C})$ eine Hermitesche Matrix, zeigen Sie, dass die folgenden Aussagen "aquivalent sind:
 \begin{enumerate}
 \item $M$ ist positiv semidefinit.
 \item Es gibt eine Matrix $W \in \text{Mat}(n\times n;\mathbb{C})$ f"ur welche $M = \overline{\transpose \ W} W$.
 \end{enumerate}
 \end{aufgabe}


% \footnotetext{*: Ggf.\ im Tutorium besprechen.}


\end{document}

%%% Local Variables: 
%%% mode: latex
%%% End: 
